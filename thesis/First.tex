\documentclass[a4paper, 11pt]{article}
\usepackage[pdftex]{graphicx}
\usepackage{fullpage}
\usepackage{mathrsfs,amsmath}
\usepackage{framed, color, fancybox}
\usepackage{todonotes}

\definecolor{shadecolor}{rgb}{0.8, 0.8, 0.2}
%define the title
\author{Seogi Kang}
\title{On recovering of distributed IP parameters in time domain electromagnetic data}

% My equations
%-----------------------------------------------------------
\renewcommand{\div}{\nabla\cdot}
\newcommand{\grad}{\vec \nabla}
\newcommand{\curl}{{\vec \nabla}\times}
\newcommand {\J}{{\vec J}}
\renewcommand{\H}{{\vec H}}
\newcommand {\E}{{\vec E}}
\newcommand{\siginf}{\sigma_\infty}
\newcommand{\dsig}{\triangle\sigma}
\newcommand{\dcurl}{{\mathbf C}}
\newcommand{\dgrad}{{\mathbf G}}
\newcommand{\Acf}{{\mathbf A_c^f}}
\newcommand{\Ace}{{\mathbf A_c^e}}
\renewcommand{\S}{{\mathbf \Sigma}}
\newcommand{\St}{{\mathbf \Sigma_\tau}}
\newcommand{\T}{{\mathbf T}}
\newcommand{\Tt}{{\mathbf T_\tau}}
\newcommand{\diag}{\mathbf{diag}}
\newcommand{\M}{{\mathbf M}}
\newcommand{\MfMui}{{\M^f_{\mu^{-1}}}}
\newcommand{\MfMuoi}{{\M^f_{\mu_0^{-1}}}}
\newcommand{\dMfMuI}{{d_m (\M^f_{\mu^{-1}})^{-1}}}
\newcommand{\dMfMuoI}{{d_m (\M^f_{\mu_0^{-1}})^{-1}}}
\newcommand{\MeSig}{{\M^e_\sigma}}
\newcommand{\MeSigInf}{{\M^e_{\sigma_\infty}}}
\newcommand{\MeSigInfEtab}{{\M^e_{\sigma_\infty \bar{\eta}}}}
\newcommand{\MeSigInfEtat}{{\M^e_{\sigma_\infty \peta}}}
\newcommand{\MedSig}{{\M^e_{\triangle\sigma}}}
\newcommand{\MeSigO}{{\M^e_{\sigma_0}}}
\newcommand{\Me}{{\M^e}}
\newcommand{\Js}{\mathbf{J}^s}
\newcommand{\Mes}[1]{{\M^e_{#1}}}
\newcommand{\Mee}{{\M^e_e}}
\newcommand{\Mej}{{\M^e_j}}
\newcommand{\BigO}[1]{\mathcal{O}\bigl(#1\bigr)}
\newcommand{\bE}{\mathbf{E}}
\newcommand{\bEp}{\mathbf{E}^p}
\newcommand{\bB}{\mathbf{B}}
\newcommand{\bBp}{\mathbf{B}^p}
\newcommand{\bEs}{\mathbf{E}^s}
\newcommand{\bBs}{\mathbf{B}^s}
\newcommand{\bH}{\mathbf{H}}
\newcommand{\B}{\vec{B}}
\newcommand{\D}{\vec{D}}
\renewcommand{\H}{\vec{H}}
\newcommand{\s}{\vec{s}}
\newcommand{\bfJ}{\bf{J}}
\newcommand{\vecm}{\vec m}
\renewcommand{\Re}{\mathsf{Re}}
\renewcommand{\Im}{\mathsf{Im}}
\renewcommand {\j}  { {\vec j} }
\newcommand {\h}  { {\vec h} }
\renewcommand {\b}  { {\vec b} }
\newcommand {\e}  { {\vec e} }
\renewcommand {\d}  { {\vec d} }
\renewcommand {\u}  { {\vec u} }

\renewcommand {\dj}  { {\mathbf{j} } }
\renewcommand {\dh}  { {\mathbf{h} } }
\newcommand {\db}  { {\mathbf{b} } }
\newcommand {\de}  { {\mathbf{e} } }

\newcommand{\vol}{\mathbf{v}}
\newcommand{\I}{\vec{I}}
\newcommand{\A}{\mathbf{A}}
\newcommand{\bI}{\mathbf{I}}
\newcommand{\bus}{\mathbf{u}^s}
\newcommand{\brhss}{\mathbf{rhs}_s}
\newcommand{\bup}{\mathbf{u}^p}
\newcommand{\brhs}{\mathbf{rhs}}
%%-------------------------------
\newcommand{\bon}{b^{on}(t)}
\newcommand{\bp}{b^{p}}
\newcommand{\dbondt}{\frac{db^{on}(t)}{dt}}
\newcommand{\dfdt}{\frac{df(t)}{dt}}
\newcommand{\dfdtdsiginf}{\frac{\partial\frac{df(t)}{dt}}{\partial\siginf}}
\newcommand{\dfdsiginf}{\frac{\partial f(t)}{\partial\siginf}}
\newcommand{\dbgdsiginf}{\frac{\partial b^{Impulse}(t)}{\partial\siginf}}
\newcommand{\digint}{\frac{2}{\pi}\int_0^{\infty}}
\newcommand{\Gbiot}{\mathbf{G}_{Biot}}
%%-------------------------------
\newcommand{\peta}{\tilde{\eta}}
\newcommand{\eFmax}{\e^{F}_{max}}
\newcommand{\dip}{d^{IP}}
\newcommand{\sigpert}{\delta\sigma}

\begin{document}
%generate the title

\maketitle

\tableofcontents
\clearpage
%%% ===========================================================================
%%% SECTION 1.
%%% ===========================================================================
\section{Introduction}

%%% ===========================================================================
%%% SECTION 2. FIGURES
%%% ===========================================================================
\section{Backgrounds}

%%% ===========================================================================
%%% SECTION 2.1. MAIN
\subsection{Complex conductivity}
An often-used representation for complex conductivity in the frequency domain is the Cole-Cole model \cite{COLE}:
\begin{equation}
  \sigma(\omega) = \sigma_{\infty} - \sigma_{\infty}\frac{\eta}{1+(1-\eta)(\imath\omega\tau)^c} = \sigma_{\infty} + \triangle\sigma(\omega),
  \label{eq: sigma_freq}
\end{equation}
where $\sigma_{\infty}$ is the conductivity at infinite frequency, $\eta$ is the intrinsic chareability, $\tau$ is the time constant and $c$ is the frequency dependency. Real and imaginary parts of complex conductivity in frequency domain are shown in Figure ~\ref{Fig:FDandTDCole}(a) with Cole-Cole parameters: $\siginf = ?$, $\eta = ?$, $\tau = ?$ and $c=1$. By applying inverse Fourier transform with time dependency, $e^{\imath\omega t}$, we have
\begin{equation}
  \sigma(t) = \mathscr{F}^{-1}[\sigma(\omega)] = \sigma_{\infty}\delta(t) + \triangle\sigma(t)
  \label{eq: sigma_time}
\end{equation}
where $\delta(t)$ is Dirac delta function and $\mathscr{F}^{-1}[\cdot]$ is inverse Fourier transform operator. Computation of $\triangle\sigma(t)$ can be convenient by assuming $c=1$, which is usually called Debye model. By evaluating inverse Fourier transform of $\triangle\sigma(\omega)$ when $c=1$, we have
\begin{equation}
  \sigma(t) = \sigma_{\infty}\delta(t) - \siginf\peta^{I}(t),
  \label{eq: sigma_time_c1}
\end{equation}
where intrinsic pseudo-chargeability, $\peta^{I}(t)$ is
\begin{equation}
    \peta^{I}(t) = \frac{\eta}{(1-\eta)\tau}e^{-\frac{t}{(1-\eta)\tau}}u(t)
    \label{eq: intrinsic_peta}
\end{equation}
and $u(t)$ is Heaviside step function. This explicit expression for time domain Cole-Cole model can give us some insights about IP effect in time domain. Cole-Cole model in time domain is also shown in Figure ~\ref{Fig:FDandTDCole}. Used Cole-Cole parameters here are same as the above.
%%% ===========================================================================
%%% SECTION 2.1. FIGURE
\begin{figure}[htb]
  \centering
  \includegraphics[width=1.0\textwidth]{figures/FDandTDCole.png}
  \caption{Cole-Cole model in frequency domain (a) and time (b) domain. }
  \label{Fig:FDandTDCole}
\end{figure}
\clearpage

%%% ===========================================================================
%%% SECTION 2.2. MAIN
\subsection{Convolution approach}
Consider Maxwell's equations in time domain:
\begin{equation}
  \curl{\e} = -\frac{\partial \b}{\partial t}
  \label{eq: total_farad}
\end{equation}
\begin{equation}
  \curl{\frac{1}{\mu}\b} - \j= \j_{s},
  \label{eq: total_coulomb}
\end{equation}
where $\e$ is the electric field ($V/m$), $\b$ is the magnetic flux density ($Wb/m^2$) and $\mu$ is the magnetic permeability ($H/m$). Here $\j$ is the conduction current. In the frequency domain the current density $\J$ is related to conductivity via $\J(\omega) = \sigma(\omega)\E(\omega)$ where $\E$ is the electric field. Substituting equation (\ref{eq: sigma_freq}) yields:
\begin{equation}
	\J(\omega) = \siginf\E(\omega)+\triangle\sigma(\omega)\E(\omega) =\siginf\E(\omega)+\vec{J}^{pol}(\omega),
\end{equation}
where $\J^{pol}(\omega)$ is the polarization current.
Converting this relationship to time domain using a Fourier transform yields:
\begin{equation}
	\j(t) = \sigma(t)\otimes \e(t),
	\label{eq: ohmslaw1}
\end{equation}
where $\otimes$ indicates time convolution. For causal signals, which is defined when $t \ge 0$, convolution between $a(t)$ and $b(t)$ can be written as
\begin{equation}
	a(t) \otimes b(t) = \int_0^t a(u) b(t-u) du.
	\label{eq: convolution}
\end{equation}
That is the current density depends upon the previous history of the electric field. Substituting equation (\ref{eq: sigma_time}) yields:
\begin{equation}
	\j(t) = \siginf\e(t) + \dsig(t)\otimes\e(t) = \siginf\e(t) + \j^{pol}(t),
	\label{eq: ohmslaw2}
\end{equation}
where $\j^{pol}(t) = \dsig(t)\otimes\e(t)$.

As an example, we consider a chargeable block with the complex conductivity in Figure ~\ref{Fig:ChargeableBlock}, that is subjected to a constant electric field, $\e^{ss}$ at t = 0. The superscript $ss$ stands for steady-state. We assume the electric field does not change due to a chargeable body thus, $\e(t) = \e^{ss}u(t)$. Following \cite{Smith1988a}, the first term in equation (\ref{eq: ohmslaw2}) can considered as the current density if there were no polarization effects: fundamental current, $\j^{F}$. By using Cole-Cole model with $c=1$  and evaluating convolution in the polarization current density, we have
\begin{equation}
	\j^{pol}(t) = -\eta\siginf [1-e^{-\frac{t}{(1-\eta)\tau}}]\e^{ss} = -\peta(t)\siginf\e^{ss}
	\label{eq: IPdensity}
\end{equation}
with the final value at $t = \infty$ yielding $\j^{pol}=-\eta\siginf\e^{ss}=-\eta\j^F$ thus, $\eta = -\frac{\j^{pol}}{\j^{ss}}$, where $\j^{ss}=\siginf \e^{ss}$. That is, the chargeability can be considered as a fraction of the fundamental and polarization current at infinite time when the system reaches to the steady-state for the polarization charge builud up. However on the other times, this fraction changes in time. In general case we have total current:
\begin{equation}
	\j(t) = \j^{F}(t) + \j^{pol}(t) = (1-\peta(t))\j^{ss},
  \label{eq: ohmslaw3}
\end{equation}
where pseudo-chargeability is defined as 
\begin{equation}
  \peta(t) = -\frac{\j^{pol}}{\j^{ss}} 
  \label{eq: pseudochargeability_ss}
\end{equation}
The pseudo-chargeability changes in time end values at t = 0 and $\infty$ are 0 and $\eta$, respectively. Here $\sigma_0=(1-\eta)\siginf$ indicates conductivity at zero frequnecy.  Figure ~\ref{Fig:Convolution_es} shows the convolution between $\sigma(t)$ and $\e(t)$. Due to the polarization currents, total current decreases after the switch on, and it reaches to the steady state at certain late time. 

Based on this analysis, we can obtain some insights about the conventional linearization in electrical field IP (EIP) and magnetic field IP (MIP) cases (\cite{seigel1959,seigel1974}), which uses both ends of complex conductivity in frequency domain: $\sigma_0$ and $\siginf$. We can define an effective conductivity as
\begin{equation}
	\sigma_{eff}(t) = \frac{\j(t)}{\e^{ss}} = \siginf(1-\peta(t)) = \siginf + \delta\sigma(t).
  \label{eq: sigeff}
\end{equation}
For $t \rightarrow \infty$, $\peta\rightarrow \eta$ (the intrinsic chargeability) and $\sigma_{eff} = \sigma_0 = \siginf(1-\eta)$, which is well used result.
Conversely, based on equation (\ref{eq: sigeff}), we define an perturbed conductivity as
\begin{equation}
  \delta\sigma(t) = -\siginf\peta(t) =\frac{\dsig(t)\otimes \e(t)}{\e^{ss}}.
  \label{eq: sigpert}
\end{equation}
Then we can rewrite equation (\ref{eq: ohmslaw3}) as
\begin{equation}
  \j(t) = (\siginf + \delta\sigma(t))\e(t).
\end{equation}
This is a principal statement of the linearization in time domain EIP and MIP responses, since it allows us to expand Maxwell's operator in terms of $\sigpert(t)$ for each time channel. Although the assumption about constant electric field allows us to have some insights about polarization due to a chargeable body; however, it does not make sense in general cases where the electric field changes due to the chargeable body. Therefore, this assumption should be released, and will be treated later.

%%% ===========================================================================
%%% SECTION 2.1. FIGURES
\begin{figure}[htb]
  \centering
  \includegraphics[width=0.6\textwidth]{figures/ChargeableBlock.png}
  \caption{A chargeable block with the complex conductivity, which is subjected to a constant electric field, $\e^{ss}$.}
  \label{Fig:ChargeableBlock}
\end{figure}
\begin{figure}[htb]
  \centering
  \includegraphics[width=1.0\textwidth]{figures/Convolution_es.png}
  \caption{Convolution of time dependent conductivity ($\sigma(t)$) and step-on electric field in $x$-direction ($e^{ss}_x$). }
  \label{Fig:Convolution_es}
\end{figure}

\clearpage
%%% ===========================================================================
%%% SECTION 2.2. MAIN
\subsection{Decomposition of EM responses}
The basic ideas and notation presented thus far form the foundation of more general cases for grounded and airborne IP measurements. The difference arises because diffusion of the fields into the subsurface need to be taken into account and also electric fields due to polarization can become important. As in \cite{Smith1988a}, we let $\e = \e^{F} + \e^{IP}$, $\b = \b^{F} + \b^{IP}$ and $\j = \j^{F} + \j^{IP}$, where superscript $F$ indicates fundamental and $IP$ is induced polarization.
Substituting into equations (\ref{eq: total_farad}) and (\ref{eq: total_coulomb}) yields the following sequences:
\begin{equation}
  \curl({\e^{F}+\e^{IP}}) = -\frac{\partial}{\partial t} (\b^{F}+\b^{IP}),
\end{equation}
\begin{equation}
  \curl\frac{1}{\mu}(\b^{F}+\b^{IP}) - (\j^{F}+\j^{IP})= \j_{s}.
\end{equation}
By canceling out vectors associated with $EM$ terms, we have
\begin{equation}
  \curl \e^{IP} = -\frac{\partial \b^{IP}}{\partial t},
  \label{eq: eq_secondary_farad}
\end{equation}
\begin{equation}
  \curl{\frac{1}{\mu}\b^{IP}} = \j^{IP}.
  \label{eq: eq_secondary_coulomb}
\end{equation}
In addition, associated $EM$ equations can be written as
\begin{equation}
  \curl \e^{F} = -\frac{\partial \b^{F}}{\partial t},
  \label{eq: eq_primary_farad}
\end{equation}
\begin{equation}
  \curl{\frac{1}{\mu}\b^{F}} -\j^{F} = \j_s.
  \label{eq: eq_primary_coulomb}
\end{equation}
Here
\begin{equation}
  \j^{F} = \siginf\e^{F},
  \label{eq: jF}
\end{equation}
\begin{equation}
  \j^{IP} = \siginf\e^{IP} + \j^{pol}.
  \label{eq: jIP}
\end{equation}
Note that IP current density, $\j^{IP}$ is summation of the polarization current and $\siginf\e^{IP}$. Previously in convolution approach, we did not have this term because we assumed constant electric field ($\e(t) = \e^{ss}u(t)$). However, in general case, we need to consider this extra term due to polarization charge build up.

Let $F[\cdot]$ denote operator associated with Maxwell’s equations and let $d$ denote any electromagnetic field that we observe thus, this incldues both EM and IP effects. Keeping same notation, we also write $d = d^{F} + \dip$. Therefore, we define IP datum as
\begin{equation}
	\dip = d - d^{F} = F[\siginf\delta(t)+\dsig(t)]-F[\siginf\delta(t)].
    \label{eq: IPdatum_syn}
\end{equation}
This subtraction process acts as an EM decoupling process, which reduces the EM effects in the measured responses to better recognize IP effect. This formed the basics of work by \cite{routh2001}. Thus, assuming that we have a reasonable estimation for the distribution of $\siginf$ in 3D space, we can identify  IP datum, which are embedded in the measured responses. 
\bigskip
\todo[inline, color=yellow!40]{
\noindent\textbf{Questions 1: Is it true that we have only IP effect with this subtraction process?} ç\newline
Doug: We need to show under what circumstances this is true. It is tied in with the statement that $\curl\e^{IP} = 0$. i.e. the polarization decays do not cause EM induction. Handling this matter here simplified much of the following paper.\medskip\newline
\textcolor{blue}{Answer: I think that EM induction due to polarization decays are also IP effect, which is due to time dependent conductivity.}
}
\bigskip
\todo[inline, color=green!20]{
\noindent\textbf{Issue1: Naming issues of $\vec{p}$ and $\j^{IP}$. How about $\j^{pol}$?}
Solved: we use $\j^{pol}$
}
%%% ===========================================================================
%%% SECTION 2.4. MAIN
\subsection{Thoughts on IP currents}
With consideration of time dependent conductivity, current density was written as
\begin{eqnarray*}
    \j(t) = \siginf\e(t)+\j^{pol}(t),
\end{eqnarray*}
where $\j^{pol}(t)=\dsig(t)\otimes\e(t)$. Or this can be rewritten as
\begin{equation*}
    \j(t) = \j^{F}(t) + \j^{IP}(t),
\end{equation*}
where $\j^{F}(t)=\siginf\e^{F}(t)$ and
\begin{eqnarray}
		\j^{IP}(t) = \siginf\e^{IP}(t) + \j^{pol}(t).
        \label{eq: jip_with_ja}
\end{eqnarray}
We first consider anomalous current density $\j_a$. With secondary field formulation shown in equations (\ref{eq: eq_secondary_farad}) and (\ref{eq: eq_secondary_coulomb}), this can be considered as source term. Assuming no EM induction effect, we can have integral equation forms:
\begin{equation}
    \e^{IP}(\vec{r}; t) = \int_{\Omega} \bar{\mathbf{G}}^{E}(\vec{r}, \vec{r}_s)\otimes\j^{pol}(\vec{r}_s; t) d\vec{r}_s,
\end{equation}
\begin{equation}
    \b^{IP}(\vec{r}; t) = \int_{\Omega} \bar{\mathbf{G}}^{B}(\vec{r}, \vec{r}_s)\otimes\j^{pol}(\vec{r}_s; t) d\vec{r}_s,
\end{equation}
where $\bar{\mathbf{G}}^{E}$ and $\bar{\mathbf{G}}^{B}$ are electric and magnetic green's tensors. Assuming we know $\j^{pol}$, we can compute $\e^{IP}$ and $\b^{IP}$ by evaluating those integrals. This represents importance of  $\j^{pol}$ for IP responses. In convolution approach $\j^{pol}$ was defined as
\begin{equation}
    \j^{pol}(t) = -\peta(t)\siginf\e^{ss}=-\peta(t)\j^{\ ref},
\end{equation}
where the reference current density is $\j^{\ ref} = \j^{ss} = \siginf\e^{ss}$.
This allows to build up a conceptual model of IP responses: IP body acts like a dipole which has opposite direction to reference current density, and proportional to $\peta$. This conceptual model is analogous to Seigel's one (\cite{seigel1959}).

Second, we treat IP current density $\j^{IP}$. Based on equation (\ref{eq: eq_secondary_farad}), we can apply Biot-Savart law:
\begin{equation}
  \b^{IP}(\vec{r}; t) = \frac{\mu_0}{4\pi}\int_{\Omega}  \frac{\j^{IP}(\vec{r}_s; t)\times\hat{r}}{|\vec{r}-\vec{r}_s|^2}d\vec{r}_s,
\end{equation}
thus we can compute $\b^{IP}$ using $\j^{IP}$. Interestingly, this shows we can compute $\b^{IP}$, once we know $\j^{IP}$. Different from magnetic greens tensor, $\bar{\mathbf{G}}^B$, a kernel in Biot-Savart law is purely geometric. However we need to know additional term $\siginf\e^{IP}$ to compute $\b^{IP}$ using Biot-Savart law. This is caused by difference between $\bar{\mathbf{G}}^B$ and kernel function of Biot-Savart law. Both approaches can be useful in spite of some different features. These two kernels can be same for a specific case where we do not have any conductivity contrast in $\siginf$. Biot-savart law can have different form (CITE):
\begin{equation}
  \b^{IP}(\vec{r}; t) = \frac{\mu_0}{4\pi}\int_{\Omega}  \frac{\vec{\nabla}_s \times \j^{IP}(\vec{r}_s; t)}{|\vec{r}-\vec{r}_s|}d\vec{r}_s.
  \label{eq: Biot}
\end{equation}
If we let $\e^{IP} = -\grad\phi^{IP}$ and $\siginf$ is constant, then we can substitute $\j^{IP}$ in above equation as $\j^{pol}$ because $\curl \grad \phi^{IP} = 0$. This result shows that we do not need to know $\siginf \e^{IP}$ term to compute $\b^{IP}$ for this specific case. 

We have recognized that $\j^{IP}$ can be a fundamental element of understanding IP responses. In order to get some insights of $\j^{IP}$, we decompose $\j^{IP}$ as
\begin{equation}
    \j^{IP}(t) = \siginf\e^{IP}(t) + \dsig(t)\otimes\e^{F}(t)+ \dsig(t)\otimes\e^{IP}(t).
    \label{eq: jip_three}
\end{equation}
For notational convenience, we respectively refer to $\siginf\e^{IP}(t)$, $\dsig(t)\otimes\e^{F}(t)$ and $\dsig(t)\otimes\e^{IP}(t)$ as $\j^{IP}_1$, $\j^{IP}_2$ and $\j^{IP}_3$. Characteristics of each current density are following. We start with $\j^{IP}_1$. This is defined everywhere. This is the term makes a difference between $\j^{IP}$ and $\j^{pol}$. Recalling Smith’s approximate convolution approach (\cite{Smith1988a}), he assumed $\j^{IP}_1$ and $\j^{IP}_3$ are negligible, which makes $\j^{IP} = \j^{pol}$. In his case, a IP body was surrounded by the free space and $\eta \ll 1$.  However, in general case where the background medium is not free space, this term is not negligible in that $\j^{IP}_2$ and $\j^{IP}_3$ are only defined in the IP body because $\dsig(t)$ is zero except for IP body. This shows that $\j^{IP}_2$ is going to be the main term of $\j^{IP}$ in the IP body as the conductivity of background medium and $\eta$ decrease. In addition, $\j^{IP}_3$ might be negligible when we have small $\eta$. However, it is hard to suggest that when we have considerable $\eta$, since this is convoluted property of $\e^{IP}$ and $\dsig(t)$.

%%% ===========================================================================
%%% SECTION 3. MAIN
%%% ===========================================================================
\section{Linearization of IP responses}
%%% ===========================================================================
%%% SECTION 3.1. MAIN
\subsection{Linearization in EIP and MIP}
For our linearization it is useful to think in terms of EIP or MIP experiment where we apply a direct current into the ground. We use step-on waveform. Neglecting induction, $F[\siginf]=F_{DC}[\siginf] \rightarrow \e^{ss}_{\siginf}$, where $F_{DC}[\cdot]$ indicates steady-state Maxwell's operator. Recalling the definitions of $\delta\sigma$ and $\sigma_{eff}$ that we have made in convolution approach, we approximately rewrite total current density, $\j$ as
\begin{eqnarray}
	\j(t) \approx (\siginf + \delta\sigma(t))\e(t) = \sigma_{eff}(t)\e(t).
  	\label{eq: ohmslaw4}
\end{eqnarray}
The perturbed conductivity and effective conductivity were defined as $\delta\sigma = \frac{\dsig(t)\otimes\e(t)}{\e^{ss}}$ and $\sigma_{eff} = \frac{\j}{\e^{ss}}$. Considering that we use step-on waveform, this is reasonable approximation, since $\e(t)\approx \e^{ss}u(t)$ when $\eta \ll 1$. By substituting $\vec{e}^{ss}$ in $\delta\sigma(t)$ with $\vec{e}(t)$, equation (\ref{eq: ohmslaw4}) can be converted to an equality equation. Based on this, by substituting constant electric field $\e^{ss}$ as $\e_{\siginf}^{ss}$, we let
\begin{equation}
  \dsig(t)\otimes\e(t) = \j^{pol}(t) \approx \delta\sigma(t)\e(t).
  \label{eq: jpol_approx}
\end{equation}
Based on this assumption for EIP problem, IP response at any time $t>0$ can be expressed as$F_{DC}[\sigma_{eff}(t)]$. Then we make Taylor expansion and obtain
\begin{equation}
  F_{DC}[\siginf + \delta\sigma_i] = F_{DC}[\siginf] + \frac{\partial F_{DC}[\siginf]}{\partial\siginf}\delta\sigma_i + O(\delta\sigma_i^2),
  \label{eq: TaylorDC}
\end{equation}
where subscript $i$ indicates $i^{th}$ time channel. By ignoring the second order term in equation (\ref{eq: TaylorDC}) with some linear algebra, therefore, we have IP datum at $i^{th}$ time channel:
\begin{equation}
  \dip_i = F_{DC}[\siginf + \delta\sigma_i] - F_{DC}[\siginf] \approx -\frac{\partial F_{DC}[\siginf]}{\partial log(\siginf)}\peta_i,
  \label{eq: EIPlinear1}
\end{equation}
where $\peta_i = -\frac{\sigpert_i}{\siginf}$. This constructs a general formulation of linear inverse problem in EIP (\cite{Yuval1997}, \cite{Hordt2006}). In addition, this still applicable for MIP by setting the type of $d$ and $F_{DC}$ as magnetic field (\cite{seigel1974}, \cite{Chen2003}).

Conversely, all of our derivations has been done for a step-on current. Then what happen if data are measured in the off-time? This can be treated simply because we can compute off-time data using on-time data, since we have relationship between them as
\begin{equation}
  \dip_{off}(t) = \dip_{on}(t=\infty) - \dip_{on}(t) = F_{DC}[\sigma_0] - \dip_{on}(t),
\end{equation}
where $\sigma_0 = \siginf(1-\eta)$, which is the low frequency limit of Cole-Cole model. This indicates that we can still use equation (\ref{eq: EIPlinear1}), because the off-time result is directly related to the on-time as illustrated in Figure ~\ref{Fig:EIPcurve}. However, in practice, the input current is not step-on or step-off but be arbitrary. We can imagine a current waveform $I_0w(t)$. The maximum charging will occur when a uniform field has been applied for a duration $t \gg \tau$. Let $\e^{ss}$ represents the steady state field associated with $I(t) = I_0u(t)$. In this case, the assumption that we made for $\j^{pol}$ (equation (\ref{eq: jpol_approx})) may not be reasonable, because the total electric field, $\e(t)$ can dynamically change due to $w(t)$ so that the electric field at $i^{th}$ time channel, $\e_i$ can be quite different from $\e^{ss}$ as shown in Figure ~\ref{Fig:Oscillating_e}. Therefore, for this general waveform, it is not straight forward to linearize EIP responses with the current framework. 

%%% ===========================================================================
%%% SECTION 3.1. FIGURE
\begin{figure}[htb]
  \centering
  \includegraphics[width=0.8\textwidth]{figures/EIPcurve.png}
  \caption{Time domain EIP curve. }
  \label{Fig:EIPcurve}
\end{figure}
\begin{figure}[htb]
  \centering
  \includegraphics[width=0.8\textwidth]{figures/Oscillating_e.png}
  \caption{Conceptual diagram of oscillating electric field.}
  \label{Fig:Oscillating_e}
\end{figure}
\clearpage

%%% ===========================================================================
%%% SECTION 3.2. MAIN
\subsection{Linearization in EMIP}
In previous two sections, we have treated linearizing IP responses without any EM induction effect. The nice property of EIP and MIP cases was our fundamental responses are time independent, thus, any time dependent responses are due to IP effect. There we assumed that $\frac{\partial \b}{\partial t}\approx 0$. However, in some situations this may not be a reasonable approximation; inductive source TEM can be an extreme example. Therefore, we need to release this assumption to handle those situations where we cannot ignore EM induction effect. Now a principal difference from steady-state IP case is not only fields due to IP effect but also the fundamental fields can be time dependent. However, still, we may want to use similar approach from steady-state case, which expands maxwell's operator in terms of $\sigpert(t)$. A significant problem here is EM induction term ($\frac{\partial \b}{\partial t}$) in equation (\ref{eq: total_farad}). This term makes a coupling between your current and previous time channels, thus expanding this operator in terms of $\sigpert(t)$:
\begin{equation*}
  F[\siginf+\dsig(t)] \approx F[\siginf] + \frac{\partial F[\siginf]}{\partial \siginf}\sigpert
\end{equation*}
is not valid. Accordingly, we need a different approach to tackle this problem, although most of fundamental methodology used in previous cases can be similar in this case.

Nonetheless, our goal is stil quite same as before: we want to express $d^{IP}$ as a function of the pseudo-chargeability $(\peta(t))$ in time. Therefore, we can have a form of linear equation like $\dip(t) = -J[\peta(t)]$, where $J[\cdot]$ is a linear operator, which is independent on time. For this, we consider a general system whether we can have either grounded or inductive source and any types of measured field. Let total electric field, $\e(t)$ can be approximated as
\begin{equation}
  \e(t) \approx \e^{\ ref}w^e(t),
  \label{eq: e_with_eref}
\end{equation}
where $\e^{\ ref}$ is the reference electric field and $w^e(t)$ is defined as:
% \begin{equation}
%   w^e(t) = \frac{\e(t)\cdot\e^{\ ref}}{\e^{\ ref}\cdot\e^{\ ref}}.
%   \label{eq: we}
% \end{equation}
\begin{equation}
  w^e(t) = \left\{ 
  \begin{array}{l l}
    w^{ref}(t) & w^{ref}(t) \ge 0 \\
    0 & \text{if } w^{ref}(t) < 0, 
  \end{array}\right.
  \label{eq: we}
\end{equation}
with
\begin{equation}
  w^{ref}(t) = \frac{\e(t)\cdot\e^{\ ref}}{\e^{\ ref}\cdot\e^{\ ref}}.
  % \label{eq: we}
\end{equation}
Here $w_{ref}(t)$ is a dimensionless function that prescribes the time history of the electric field at each location. $\e^{\ ref}$ is an appropriate scaling for the electric field, which is independent on time. 
This indicates that the direction of the electric field does not change in time. 
In addition, by the definition of $w^e(t)$, we projects negatives values of  $w^{ref}(t)$ to zero.
Note that this assumption is only applied to the second term on right-hand side of equation (\ref{eq: jIP}). 
Substituting this into equation (\ref{eq: jIP}) yields
we have
\begin{eqnarray*}
  \j^{IP}(t) \approx \siginf\e^{IP}(t) + \frac{\dsig(t)\otimes w^e(t)}{\siginf}\j^{\ ref},
\end{eqnarray*}
where the reference current density is defined as
\begin{equation}
  \j^{ref} = \siginf\e^{\ ref}.
\end{equation}
Letting pseudo-chargeability as
\begin{equation}
    \peta(t) = -\frac{\dsig(t)\otimes w^e(t)}{\siginf} \approx -\frac{\j^{pol}}{\j^{\ ref}},
    \label{eq: pseudochargeability}
\end{equation}
yields
\begin{eqnarray}
  \j^{IP}(t) = \siginf\e^{IP}(t) + \j^{pol}(t) \approx \siginf\e^{IP}(t) -\j^{\ ref}\peta(t).
  \label{eq: jip_EMIP}
\end{eqnarray}
A physical insight about pseudo-chargeability is that a fraction of polarization current and reference current. Because the reference current is time-independent property, any time-dependency in the pseudo-chargeability is from polarization current. Using Helmholtz decomposition, $\e$ can be decomposed as $\e=-\vec{a}-\grad\phi$, where $\vec{a}$ and $\phi$ is electric vector and scalar potentials, respectively and $\div\vec{a}=0$. Physically, those two terms indicate charge-build up and EM induction effects, which induce galvanic and vortex currents, respectively. So, $\e^{IP}$ can be decomposed as
\begin{equation}
  \e^{IP}=-\vec{a}^{IP}-\grad\phi^{IP}.
\end{equation}
Now we make an another assumption:
\begin{equation}
  \e^{IP} \approx  \e^{IP}_{approx} = -\grad\phi^{IP},
  \label{eq: eip_approx}
\end{equation}
which means $\e^{IP}$ term in equation (\ref{eq: jip_EMIP}) is dominated by galvanic effect ($\frac{\partial\b^{IP}}{\partial t}\approx 0$). First taking $\div$ to equation (\ref{eq: jip_EMIP}) then by substituting  $\e^{IP}$ with equation (\ref{eq: eip_approx}) with some linear algebra, we obtain
\begin{equation}
  \phi^{IP}(t) \approx -[\div \siginf\grad]^{-1}\div\j^{\ ref}\peta(t).
  \label{eq: phiIPapprox_general}
\end{equation}
By taking $\grad$ to this equation we have
\begin{equation}
    \e^{IP}_{approx} = \grad[\div \siginf\grad]^{-1}\div\j^{\ ref}\peta(t).
    \label{eq: eip_approx_full}
\end{equation}
Thus, we can compute $\e^{IP}_{approx}$, and thus electric field due to IP effect can be expressed as a function of $\peta(t)$ in time.

Some EM instruments measure $\b$ or $\frac{\partial \b}{\partial t}$, so naturally we want to compute $\b^{IP}$ or $\frac{\partial \b^{IP}}{\partial t}$. For this, first we compute $\j^{IP}$ then use Biot-Savart law to compute $\b^{IP}$ or $\frac{\partial \b^{IP}}{\partial t}$. Substituting equation (\ref{eq: phiIPapprox_general}) into equation (\ref{eq: jip_EMIP}), approximated IP current density, $\j^{IP}_{approx}$ can be expressed as
\begin{equation}
  \j^{IP}(t) \approx \j^{IP}_{approx} = -\bar{S}\siginf\e^{\ ref}\peta(t),
  \label{eq: jip_approx}
\end{equation}
where
\begin{equation}
  \bar{S} = -\siginf\grad[\div \siginf\grad]^{-1}\div+\bar{I}
\end{equation}
and $\bar{I}$ is an identity tensor. Then by using Biot-Savart law we have:
\begin{equation}
  \b^{IP}_{approx}(\vec{r}; t) = -\frac{\mu_0}{4\pi}\int_{\Omega}  \frac{\bar{S}\e^{\ ref}(\vec{r}_s)\times\hat{r}}{|\vec{r}-\vec{r}_s|^2}\peta(t)d\vec{r}_s.
  \label{eq: BiotbIP_approx}
\end{equation}
By taking time derivative to the above equation we have
\begin{equation}
  \frac{\partial\b^{IP}_{approx}}{\partial t}(\vec{r}) = -\frac{\mu_0}{4\pi} \int_{\Omega}  \frac{\bar{S}\e^{\ ref}(\vec{r}_s)\times\hat{r}}{|\vec{r}-\vec{r}_s|^2}\frac{\partial}{\partial t}\peta(t)d\vec{r}_s.
  \label{eq: BiotbIPdt_approx}
\end{equation}
Both equations (\ref{eq: BiotbIP_approx}) and (\ref{eq: BiotbIPdt_approx}) are function of $\peta(t)$, and the time dependency only rises from $\peta(t)$. 
To summarize, with the assumption that $\e^{IP}$ is mostly galvanic, we show that electromagnetic fields due to IP effects can be expressed as instantaneous property, which have linear relationship with $\peta(t)$. Therefore, $d^{IP}$ responses, which can be any type of electromagnetic fields, can be represented as $\dip(t) = -J[\peta(t)]$. In discretized space this can be:
\begin{equation}
  \mathbf{d}^{IP}_i = -\mathbf{J}\peta_i,
  \label{eq: dIP_lineareq}
\end{equation}
where $\mathbf{J}$ is corresponding sensitivity matrix and the subscript $i$ indicates $i^{th}$ time channel. Boldface of upper and lower cases indicate a matrix and column vector in discretized space. Since $d^{IP}$ can be any types of electromagnetic fields, our approach to linearize time domain IP responses is applicable for any types of TEM surveys, whereas some details for practical application might be somewhat different. However, the fundamental concept will be same.

On the other hand, we must carefully investigate the assumptions that we have made to formulate this linear relationship, and a fundamental characteristics of what we are recovering; thus, we need to identify four fundamental points. We made two assumptions: (a) we know exact background conductivity model ($\siginf$). (b) Inductive effect in $\e^{IP}$ is negligible. (c) $\dsig(t)\otimes \e(t) = -\peta(t)\j^{\ ref}$, In addition, (d) we recognized that $\peta(t)$ is convoluted property with $\e(t)$ and distributed IP parameters. Those four points are not trivial to be validated in physical or mathematical ways; thus each of them should be carefully tested with numerical experiments for different types TEM surveys like grounded or inductive source TEM methods and different conductivity structures.

%%% ===========================================================================
%%% SECTION 3.4. MAIN
\subsection{Revisitation of EIP and MIP}
In previous section, we suggested a linearization methodology, which can be applied to general TEM problems. Using this methodology, we revisit linearization of IP responses in EIP and MIP cases. Recall that we had a challenge in linearization when we have general current waveform.

A proper reference electric field, $\e^{\ ref}$ for EIP and MIP cases can be $\e^{ss}_{\siginf}$, so we modify equation (\ref{eq: e_with_eref}) as
\begin{equation*}
    \e(t) \approx \e^{ss}_{\siginf}w^e(t).
\end{equation*}
Then $\phi^{IP}(t)$ shown in equation (\ref{eq: phiIPapprox_general}), can be rewritten as
\begin{eqnarray*}
  \phi^{IP}(t) \approx \phi^{IP}_{approx} =  -[\div\siginf\grad]^{-1}\div\grad\phi^{ss}_{\siginf}(\dsig(t)\otimes w^e(t)) \\
               =\frac{\partial \phi^{ss}_{\siginf}}{\partial \siginf}\dsig(t)\otimes w^e(t)
\end{eqnarray*}
and finally we have
\begin{eqnarray}
    \phi^{IP}_{approx}(t) = \frac{\partial \phi^{ss}_{\siginf}}{\partial \siginf}\dsig(t)\otimes w^e(t) \nonumber\\
                 =\frac{\partial \phi^{ss}_{\siginf}}{\partial \siginf}\sigpert(t)
\end{eqnarray}
where $\sigpert(t) = \frac{\dsig(t)\otimes \e(t)}{\e^{ss}_{\siginf}}=\dsig(t)\otimes w^e(t)$ and
\begin{equation}
    \frac{\partial \phi^{ss}_{\siginf}}{\partial \siginf} = -[\div\siginf\grad]^{-1}\div\grad\phi^{ss}_{\siginf}.
    \label{eq: senseDC}
\end{equation}
Detailed derivation of equation (\ref{eq: senseDC}) in discretized space is given in appendix ??. Substituting equation (\ref{eq: pseudochargeability}) to (\ref{eq: senseDC}), we have
\begin{equation}
    \phi^{IP}_{approx}(t) = -\frac{\partial \phi^{ss}_{\siginf}}{\partial log(\siginf)}\peta(t),
    \label{eq: phiIPapprox}
\end{equation}
which is the same result from equation (\ref{eq: EIPlinear1}) for electrical potential. Therefore, we can derive same result with conventional linearization in EIP case using suggested methodology. $\e^{IP}$ can also be linearized in the same way, since we can just take gradient to the above equation and obtain $\e^{IP}$.

We also measure magnetic flux density, $\b$, so we similarly linearize $\b^{IP}$ by expanding Maxwell's operator. However, in this time we use different method, which was suggested in previous section. Modifying equation (\ref{eq: jip_EMIP}), we have
\begin{eqnarray}
    \j^{IP}(t) = \siginf\e^{IP}(t) + \dsig(t)\otimes w^e(t)\e^{ss}_{\siginf} \nonumber \\
            = -\bar{S}\siginf\e^{ss}_{\siginf}\peta(t).
\end{eqnarray}
Applying Biot-Savart law yields
\begin{equation}
  \b^{IP}(\vec{r}; t) = -\frac{\mu_0}{4\pi}\int_{\Omega}  \frac{\bar{S}\e^{ss}_{\siginf}(\vec{r}_s)\times\hat{r}}{|\vec{r}-\vec{r}_s|^2}\peta(t)d\vec{r}_s.
  \label{eq: BiotbIP_ss}
\end{equation}
This result shows the applicability of our methodology to MIP. Different from conventional linearization approach used in EIP and MIP cases, applicability of our linearization method is not limited to step-on or step-off waveform, but general. Therefore, even when we have general waveform, which can be oscillating as shown in Figure ~\ref{Fig:Oscillating_e}, the linearzation of EIP or MIP can still proceed. Furthermore, considering an oscillating input current as shown in Figure ~\ref{Fig:Oscillating_e} also supports this statement, because $|\delta(t)| \ll \siginf\eta$.

Interestingly, except ignoring EM induction effect, there is no assumption in our linearization for these EIP and MIP cases so that we can compute exact $\d^{IP}$ responses if we have correct $\peta(t)$. However, we recognize that pseudo-chargeability, $\peta(t)$ is convoluted property between $\e(t)$ and $\dsig(t)$, that is, it is not independent material property, but coupled property with the fields. Then we need to have some insights about $\peta(t)$, since we are going to recover distribution $\peta(t)$, eventually. Assuming $\dsig(t)\otimes \e(t) \approx \dsig(t)\otimes \e^F(t)$, $\peta(t)$ can be modified as
\begin{equation}
    \peta(t) = -\frac{\dsig(t)\otimes w(t)}{\siginf},
\end{equation}
where the input current waveform: $I(t) = I_0w(t)$ and fundamental electric field, $\e^{F} = \e^{ss}_{\siginf}w(t)$. Therefore, $\peta(t)$ is not coupled with the electric field, but with the normalized waveform, $w(t)$. Therefore, pseudo-chargeability can provide meaningful information of distributed IP parameters under our assumption, whereas this assumption should be investigated carefully.
%%% ===========================================================================
%%% SECTION 3.5. MAIN
\subsection{Discussions}

\subsubsection{Choice of reference electric field}
Consider a step-on current waveform with grounded source. Different from EIP or MIP case, here we do not ignore EM induction effect, therefore, even fundamental responses are coupled with time, that is, responses at current time channel are dependent on previous time channel. In spite of this difference, initial and final responses, which respectively refer $\lim_{t\rightarrow 0}d (t)$ and $\lim_{t\rightarrow \infty}d (t)$, are going to be same as EIP case as shown in Figure (\ref{Fig:EIPcurve}). A fundamental difference may be induced by $\e^F(t)$, since it contains not only charge build-up effect, but also EM induction effect. Since both effects generates IP response, we need to identify which one is dominant term. This may not be easy, since the second term can have arbitrary time decaying feature based on the conductivity distribution of the earth. Nonetheless, if the first term is dominant for all time channels, then it is reasonable to choose $\e^{ss}_{\siginf}$ as $\e^{\ ref}$, which is same as EIP case. In contrast, if the second term is dominant, an intuitive choice can be $\e^{F}_{max} = \e^{F}(t) \otimes \delta(t-t^{max})$, where
\begin{equation}
  \e^{F}_{max} = \e^{F}(t) \otimes \delta(t-t^{max}),
\end{equation}
where $t^{max}$ is the time when the magnitude of fundamental electric fields reach to the maximum. Note that $t_{max}$ is variable in 3D space, since electric field at each pixel in 3D space can have different time decaying feature due to the source location and conductivity distribution.

A fundamental difference from galvanic and inductive source arises from the property, $\div\j_s$.  If this is zero then it is inductive source or if not, it is galvanic source. In order to get some insight, consider DC problem for inductive source:
\begin{eqnarray*}
    -\div\j^{ss} = \div\j_s = 0, \\
    \div \siginf \grad \phi^{ss}_{\siginf} = -\div\j_s = 0.
\end{eqnarray*}
Since the right-hand side of the above equation is zero, thus, $\phi^{ss}_{\siginf}$ = 0 and $\e^{ss}_{\siginf}=0$. This shows that there are no DC fields for inductive source case. Then natural choice of $\e^{\ ref}$ is going to be $\e^{F}_{max} = \e^{EM}_{max}$. Consider a step-off waveform. After the current is switched off, electric field starts to increase (may be oscillating) from zero, and reaches at the maximum, then decays. Therefore, we recognize that $\e^{F}_{max}$ as $\e^{\ ref}$ for inductive source case is reasonable.

\bigskip
\todo[inline, color=green!20]{
\noindent\textbf{Constant ramp for inductive source: obtain same analogy for grounded source case (constant electric field)}\newline\newline
Polarization charge build up reaches to the steady state
}

\subsubsection{Ignoring inductive IP effect}

In constrolled-source EM (CSEM) experiments, we can have two types of sources: grounded and inductive sources. We first consider this assumption for grounded source. Ignoring inductive part of IP effect $\frac{\partial \b^{IP}}{\partial t}\approx 0$, might be okay for grounded source case, since galvanic effect  can be dominant ($\grad\phi^{IP} \gg \vec{a}^{IP}$) in most cases, but this should be carefully tested because it depends on the conductivity structure. 

Different from galvanic source, for inductive source case, EM induction is the main driving force of the system. Therefore, we need to be more careful when we ignore inductie IP effect. Consider a situation where we have homogeneous background conductivity model ($\siginf$=constant). Recall $\j^{IP}$ with Helmholtz decomposition:
\begin{eqnarray*}
    \j^{IP} = \siginf\e^{IP} + \dsig(t)\otimes w^{e}(t)\e^{\ ref} \\
            = -\siginf(\grad\phi^{IP}+\vec{a}^{IP}) + \dsig(t)\otimes w^{e}(t)\e^{\ ref}
\end{eqnarray*}
By taking $\div$ we have
\begin{equation*}
    -\div\siginf\grad\phi^{IP}-\div\siginf\vec{a}^{IP} = -\div\dsig(t)\otimes w^{e}(t)\e^{\ ref}.
\end{equation*}
Since $\siginf$ is constant in space, then we rewrite
\begin{equation*}
    \div\siginf\grad\phi^{IP} = \j^{pol}_{approx},
\end{equation*}
where $\div\vec{a}^{IP}=0$ and $\j^{pol}_{approx}  = \div\dsig(t)\otimes w^{e}(t)\e^{\ ref}$. Therefore, once we do not have any conductivity variation in our background conductivity model, we can obtain reasonable $\phi^{IP}$ by solving above sytem. This indicates we can compute IP responses, which are related to galvanic effect, while we still can have significant induction portion, $\vec{a}^{IP}$ in $\e^{IP}$. Note that $\div\vec{a}^{IP}=0$ does not mean $\vec{a}^{IP}=0$. Further in general case where we have conductivity constrast in the earth, $\div \siginf \vec{a}^{IP}$ may not be zero, thus the relative strength of  $\siginf \vec{a}^{IP}$ to $\j^{pol}_{approx}$ should be minor to make our assumption reasonable. Intuitively, we know that EM induction effect is going to decay faster than galvanic effect so that they are going to small in late time channels. To consider this, we consider Faraday's law
\begin{eqnarray*}
    \curl \vec{a}^{IP}(t) = \frac{\partial \b^{IP}(t)}{\partial t},
\end{eqnarray*}
since $\curl \grad \phi^{IP} = 0$. This tells us time varying magnetic flux density generate rotational electric field, that is $\vec{a}^{IP}$. This induced IP effect may have an influnence on polarization charge build up ($\phi^{IP}$) due to the term $\div\siginf\vec{a}^{IP}$, when we have conductivity constrast in the earth. Relative influence of this term will vary for different conductivity structures. 
In frequency domain equation, we have
\begin{equation*}
    \curl \vec{A}^{IP}(\omega) = \imath\omega\B^{IP}(\omega).
\end{equation*}
The magnitude of $\vec{A}{IP}$ is proportional to $\omega$ in frequency domain. As $\omega$ decrease, EM induction effect gets smaller. Accordingly, $\vec{a}^{IP}$ in $\e^{IP}$ gets smaller as we move on to late time channels, thus galvanic term $\phi^{IP}$ can be dominant term in $\e^{IP}$ at certain late time channels. Those are possible situations that we are interested for IP signals and we might have some chances to apply our linearization methodology.


% which can be used in actual paper
%==============================================================================
% Due to the consideration of time dependent conductivity, current density includes convolution term (\ref{eq: ohmslaw1}), and this can make significant complexity in computing forward response of this EM system. Intuitively, inverse problem to recover distribution of time dependent conductivity can also be much more complex. Although we can directly derive inverse problem to recover 4D distribution of time dependent conductivity or distributed Cole-Cole parameters; however, we choose a linearization approach to simplify this inverse problem. Based on the experience in DC-IP inversion in  the past couple decades, we believe that the linearization approach can be a more robust solution to approach this inverse problem.

% Principal flow of linearization of time domain IP responses for inductive source has two steps: (a) The most principal component of our linearization is polarization current density, $\j^{IP}$. We linearize $\j^{IP}$ and express in terms of pseudo chargeability.  (b) Once we have $\j^{IP}$ then $\b^{IP}$ of $\frac{\partial\b^{IP}}{\partial t}$ can be computed using Biot-Savart law. Recalling that measured data for inductive source are $\b$ or $\frac{\partial\b}{\partial t}$, TEM responses can be linearized as a function of pseudo-chargeability.

% where $t^{max}$ is the time when the magnitude of fundamental electric fields reach to the maximum. Note that $t_{max}$ and $w(t)$ are variable in 3D space, since electric field at each pixel in 3D space can have different time decaying feature due to Tx-Rx configuration and background conductivity.

% Conversely, we consider that we measure $\b$ or $\frac{\partial \b}{\partial t}$ when we use inductive source for time domain EM survey. Given that we know how to compute $\j^{IP}$, we use Biot-Savart law to compute $\b^{IP}$ and $\frac{\partial\b^{IP}}{\partial t}$:

% Following the notation in \cite{Smith1988a}, we rewrite equation(\ref{eq: ohmslaw2}) as
% \begin{equation}
% 	\j = \j^{F}  + \j^{IP},
% \end{equation}
% where $\j^F$ is the fundamental current density if there were no polarization effects, and $\j^{IP}$ is the IP current density. In the same manner, $\e$ can be decomposed as $\e^{F}+\e^{IP}$, thus we have $\j^{F} = \siginf\e^{F}$ and $\j^{IP} = \siginf\e^{IP}+\dsig(t)\otimes\e(t)$.

% Therefore, in the end, by substituting equation (\ref{F: jip_approx}) to equations (\ref{F: BiotbIP}) and (\ref{F: BiotbIPdt}), we can set a linear inverse problem, which restore each time channel of $\peta(t)$, separately. For this, it is important that we need to know 3D distribution of $\siginf$, which is not trivial in practice. However, we have some chances that we can restore reasonable $\siginf$ distribution by applying 3D inversion to the measured ATEM data with exception of significantly contaminated data by IP effect (CITEyang2014).
\clearpage
%%% ===========================================================================
%%% SECTION 4. MAIN
%%% ===========================================================================
% \section{IP inversion methodology}
% \subsection{IP inversion procedure}

% Application of our linear inversion approach can be divided to main steps. First we identify $d^{IP}$ responses using equation (\ref{eq: IPdatum_syn}), then apply linear inversion to restore distribution of pseudo-chargeability. However, in practice there can be some challenges that we need to over come in both steps. Therefore, we suggest a procedure of TEM-IP inversion.

% To obtain the fundamental EM response, $d^F$, (a) invert ATEM data and recover a 3D conductivity model. This may involve omitting data that are obviously contaminated
% with IP signals, such as the existence of negative transients in coincident loop surveys. (b)Forward modelling then yields an approximate value of $d^F$ and subtract it from the observations, $d^{obs}$. However, we cannot get the correct $d^F$, since the estimated background conductivity is not exact. (c) Therefore the computed $d^{IP}$ data may contain a residual field due to the incorrect background conductivity. We assume this residual field is a large-scale smoothly varying perturbation to the $d^{IP}$ data and refer to it as regional field. To estimate it, we fit the computed $d^{IP}$ data with a low order polynomial and subtract that from the $d^{IP}$ data. This regional removal process can be expressed as
% \begin{equation}
%     d^{IP} = d^{obs} - d[\sigma_{est}] - \delta d = raw \ d^{IP} -  \delta d,
% \end{equation}
% where $\sigma_{est}$ is the background conductivity model estimated from 3D ATEM inversion, $d[\sigma_{est}]$ is the forward modelled data from $\sigma_{est}$, and $\delta d$ arises from the regional processing. Here we let $raw \ d^{IP}= d^{obs}- d[\sigma_{est}]$. (d) The final data are linearly related to a pseudo-chargeability through a sensitivity function shown in equation (\ref{eq: dIP_lineareq}). (e) The $\dip$ at various time channels can be inverted individually. The pseudo-chargeability models may be useful in themselves or they may be further processed to estimate Cole-Cole, or equivalent parameters.

% \subsection{Discretization of linear IP problem}
% In order to restore distributed $\peta(x, y, z; t)$ in 4D space, first we need to discretize Biot-Savart law shown in equations (\ref{eq: BiotbIP_approx}) and (\ref{eq: BiotbIPdt_approx}). In addition, based on the discretization that we choose, we need to compute $\e^{F}_{max}$. In our discretization $\j^{IP}$ and  $\tilde{\eta}_w$ are defined on the cell center, and those for each time channel are constant in a cell volume, whereas $\e^{F}_{max}$ is defined on the cell edges. We define the number of cells and edges in 3D space as nC and nE, respectively. Assuming we have discretized IP current densitivity distribution ($\dj^{IP}_{cc}$), which has dimension: [(3nC)$\times$1] and defined on the cell center, since $\j^{IP}$ has three components, we first discretize integration operator including cross product ($\int_{v}\frac{ \times \hat{r}}{r^2}dv$) as
% \begin{equation}
%   \mathbf{G}_{Biot} =
%   \begin{bmatrix}
%        \mathbf{e}^T &  \mathbf{0}   & \mathbf{0}  \\
%        \mathbf{0}   &  \mathbf{e}^T & \mathbf{0}  \\
%        \mathbf{0}   &  \mathbf{0}   & \mathbf{e}^T
%     \end{bmatrix}
%   \begin{bmatrix}
%        \mathbf{0}     &   \mathbf{S}_z   & -\mathbf{S}_y  \\
%       -\mathbf{S}_z   &   \mathbf{0}     &  \mathbf{S}_x  \\
%        \mathbf{S}_y   &  -\mathbf{S}_x   &  \mathbf{0}
%     \end{bmatrix},
%  \end{equation}
% where
% \begin{eqnarray*}
%   \mathbf{S}_i =\diag(\mathbf{r}_i \oplus \frac{1}{\mathbf{r}^2}), \ i = x, \ y, \ z
% \end{eqnarray*}
% and $\mathbf{e}$ is a column vector of which element is 1 and has the dimension: [nC$\times$1], $\diag(\cdot)$ is the diagonal matrix and $\oplus$ is the Hadamard product. Then we consider discrete expression for $\j^{IP}$ shown in equation (\ref{eq: jip_approx}), and modify this equation as
% \begin{eqnarray}
%   \dj^{IP}_{cc}(t) = -\mathbf{S}\diag(\de^{F}_{max})\Ace^T\diag(\vol)\diag(\siginf)\peta(t),
% \end{eqnarray}
% where
% \begin{eqnarray}
%   \mathbf{S} = \mathbf{A}^{e}_{ccv}\Me^{-1}[-\MeSigInf \mathbf{G} \A_{\siginf}^{-1}\mathbf{G}^T + \mathbf{I}] \diag(\de^{F}_{max})\Ace^T\diag(\vol)\diag(\siginf).
% \end{eqnarray}
% and $\mathbf{A}^{e}_{ccv}$ is discrete averaging matrix from edge to cell center with consideration of three component vector so that dimension of this matrix is [3nC$\times$nE]. Thus, we can have linear equation for $k^{th}$ time channel as
% \begin{eqnarray*}
%   \db^{IP}_k = -\Gbiot \mathbf{S} \peta_k,
% \end{eqnarray*}
% where sub-index $k$ indcates $k^{th}$ time channel. Finally, by letting
% \begin{equation}
%   \mathbf{J} = -\Gbiot\mathbf{S},
%   \label{eq: Sense}
% \end{equation}
% we have
% \begin{eqnarray}
%   \db^{IP}_k = -\mathbf{J}\peta_k,
%   \label{eq: bIP_linear}
% \end{eqnarray}
% where $\mathbf{J}$ is the Jacobian matrix of the linear equation, and since $\mathbf{J}$ is not time-dependent we also have
% \begin{eqnarray}
%   \frac{\partial\db^{IP}_k}{\partial t} = -\mathbf{J}\frac{\partial}{\partial t}\peta_k.
%   \label{eq: dbIPdt_linear}
% \end{eqnarray}
% Therefore, we can restore distributed $\peta(t)$ or $\frac{\partial}{\partial t}\peta(t)$ based on the data type we have. Note that problems shown in equations (\ref{eq: bIP_linear}) and (\ref{eq: dbIPdt_linear}) are linear problems; thus, by solving linear inverse problem for each time channel we can recover distributed IP parameters. Detailed description about discrete operators are shown in Appendix ??.


% \subsection{Linear inversion}
% To set up an linear inverse problem we rewrite equation (\ref{eq: dIP_lineareq}) as
% \begin{eqnarray}
%   \mathbf{d}^{pred} = \mathbf{A}\mathbf{m},
%   \label{eq9}
% \end{eqnarray}
% where $\mathbf{A}$ is sensitivity matrix of linear problem, which corresponds to $\mathbf{J}$ shown in equation (\ref{eq: dIP_lineareq}), $\mathbf{d}^{pred}$ is IP responses at $k^{th}$ time channel ($\mathbf{d}^{IP}_k$), $\mathbf{m}$ is distributed model parameters, which can be either $\peta_{k}$ or $\frac{\partial}{\partial t}\peta_{k}$. This presents that we invert each time channel of $d^{IP}$, separately. Our inversion methodology is based upon that described in \cite{doug1994}. The solution to the inverse problem is the model $\mathbf{m}$ that solves the optimization problem
% \begin{eqnarray}
%   minimize \ \phi =  \phi_d(\mathbf{m}) + \phi_m(\mathbf{m})\nonumber \\
%   s.t. \ 0 \le \mathbf{m},
%   \label{eq10}
% \end{eqnarray}
% where $\phi_d$ is a measure of data misfit, $\phi_m$ is a user defined model objective function and $\beta$ is regularization or trade-off parameter. We use the sum of the squares to measure data misfit
% \begin{eqnarray}
%   \phi_d = \| \mathbf{W_d}(\mathbf{A}\mathbf{m}-d^{obs}|)\|^2_2 =
%   \sum^N_{j=1}(\frac{\mathbf{d}^{pred}_j-\mathbf{d}^{obs}_j}{\epsilon_j}),
%   \label{eq11}
% \end{eqnarray}
% where $N$ is the number of the observed data and $\mathbf{W_d}$ is a diagonal data weighting matrix which contains the reciprocal of the esitmated uncertainty of each datum (
% $\epsilon_j$) on the main diagonal,  $\mathbf{d}^{obs}$ is a vector containing the observed data, $\mathbf{d}^{pred}$ is a vector containing calculated data from a linear equation given in equation (\ref{eq9}).
% The model objective function, $\phi_m$ is a measure of amount structure in the model and, upon minimization, will generate a smooth model which is close to a reference model, $m_{ref}$. We define $\phi_m$ as
% \begin{eqnarray}
%   \phi_m = \alpha_s\| \mathbf{W}_s\mathbf{W}(\mathbf{m}-\mathbf{m}_{ref})\|^2_2+
%        \alpha_x\| \mathbf{W}_x\mathbf{W}(\mathbf{m}-\mathbf{m}_{ref})\|^2_2+ \nonumber \\
%        \alpha_y\| \mathbf{W}_y\mathbf{W}(\mathbf{m}-\mathbf{m}_{ref})\|^2_2+
%        \alpha_z\| \mathbf{W}_z\mathbf{W}(\mathbf{m}-\mathbf{m}_{ref})\|^2_2,
%   \label{eq12}
% \end{eqnarray}
% where $\mathbf{W}_s$ is a diagonal matrix, and $\mathbf{W}_x$, $\mathbf{W}_y$ and $\mathbf{W}_z$ are discrete approximations of the first derivative operator in $x$, $y$ and $z$ directions, respectively.  The $\alpha$'s are weighting parameters that balance the relative importance of producing small or smooth models. Model weighting matrix, $\mathbf{W}$ is defined as
% \begin{equation}
%     \mathbf{W} = \mathbf{diag}(\mathbf{w}),
%     \label{eq: weight_mat}
% \end{equation}
% where $\mathbf{w}$ is a column vector for weighting each model parameters.

%% =======================================================================
%% SECTION 5. MAIN
%% =======================================================================
\section{Application: airborne time domain EM}
In order to investigate the suggested methodologies to restore distributed IP parameters from $\dip$ responses in airborne time domain EM (ATEM) problem, we compose an IP body model, which includes IP body in the half-space as shown in Figure~\ref{F: IPModel}. Cole-Cole parameters of this IP body are fixed to $\eta=0.2$, $\tau=0.005$ and $c=1$. Conductivity value of the half-space, ($\sigma_1$) is fixed to $10^{-3}$ S/m, whereas $\siginf$, which is same as $\sigma_2$ for the IP body varies. We have three different models, where we named canonical ($\sigma_2=\sigma_1$), conductive ($\sigma_2=10^2\times\sigma_1$) and resistive models ($\sigma_2=10^{-2}\times\sigma_1$).   For the discretization of 3D earth, $50\times50\times50$ m core cell is used and the number of cell in the domain is $41\times41\times40$. The size of the IP body is $250\times250\times250$ m and the top boundary of this IP body is located at $50$ m below the surface. In order to generate synthetic ATEM-IP data, we use EMTDIP code developed by \cite{Marchant2014}

Survey geometry include 11 stations in each 11 lines as shown in Figure \ref{F: IPModel}a so that we have 121 stations. We use coincident-loop system and both Tx and Rx are located 30 m above the surface; radius of the loop is 10 m. Step-off transmitter waveform is used and the range of the observed time channel is 0.01-10 ms. Measured responses can be either vertical component of $\b$ or $\frac{\partial \b}{\partial t}$ fields. Based on this set up, in this section, first we analyze ATEM responses with IP effect and identify $d^{IP}$ embedded in the observed data. Second, we validate approximations in our linearization idea by comparing $\j^{IP}$ and $\j^{IP}_{approx}$. Furthermore, for inverse problem we will not recover pseudo-chargeability for each soundings but a representative pseudo-chargeability, which is same for every sounding. We also treat this issue while we are verifying our assumptions. 
Based on this validation, third, we apply linear inversion to each time channel of $\dip$ data separately and restore 3D distribution of pseudo-chargeability for each time channel. Finally, we investigate the possibility of extracting intrinsic IP parameters from pseudo-chargeability with deconvolution approach. 

\begin{figure}[htb]
  \centering
  \includegraphics[width=0.5\textwidth]{figures/threecasesresp/Planview.png} \\
  (a) \\
  \includegraphics[width=0.5\textwidth]{figures/threecasesresp/Sectionview.png} \\
  (b)
  \caption{Planal (a) and sectional (b) views of IP model. Dashed line in (a) contours the boundary of the IP body. Solid circles in (a) denotes the location of stations.}
  \label{F: IPModel}
\end{figure}
\clearpage

\subsection{Analyses of ATEM-IP responses}
IP responses in ATEM data are different from different conductivity structure. Understanding different characteristics in IP response due to different conductivity can provide us important background physics. We use three different conductivity models: canonical, conductive and resistive. Then,  we proceed EM decoupling approach, which subtract $d^F$ from $d$,  assuming that we know true conductivity ($\siginf$), for all three cases and discuss features of IP response from different conductivity. 

For coincident loop system, the most clear IP response is negative transients. Simply, we consider that our ATEM data has IP effects when we see negative transients. For three different cases, we start with analyzing time decaying curves at the center sounding locations (0 m easting and 0 m northing). Figure \ref{F:IPrespCurves} shows $d$ (black line), $d^{F}$ (blue line) and $\dip$ (red line) at center sounding location for (a) canonical, (b) conductive and (c) resistive models. We also provided both $b_z$ and $\frac{\partial b_z}{\partial t}$ on the left and right panels, respectively. Although, we measure $\frac{\partial b_z}{\partial t}$, $b_z$ is more intuitive for us to make a connection with the IP current ($\j^{IP}$), because we can compute $b_z$ using Biot-Savart law. Polarization charge build-up may have two main phases: charging and discharging. For charging phase, IP current may increase and for discharging phase, IP current may decrease. This phenomenon will be proportional on $b^{IP}_z$ response thus, we choose a time when $b^{IP}_z$ has maximum amplitude with negative values. The reason behind we only use negative values to choose reference time is based on the background that the IP current has reversed direction to the reference current (ReferenceEq). This time may indicate the time when the earth has maximum polarization charge build up. Based on this reference time ($t^{ref}$), we decide charging and discharging phases as previous and post reference time, respectively.   We first consider canonical model ($\sigma_2 = \sigma_1$)shown in Figure \ref{F:IPrespCurves}a. In the observed data ($d$), we do not have any negative transients. However, due to the IP effect, the observed data shows rapid decay on late time channel after 1 ms. Computed $\dip$ responses (red) by subtraction of $d^F$ from $d$, have negative sign for all times, and the reference time is $~$0.1 ms. Even at this time, $d^{F}$ is far greater than $\dip$ thus, EM induction is dominant. However, at late time after 1 ms $\dip$ is considerable to $d^{F}$. $\frac{\partial b^{IP}_z}{\partial t}$ response show sign reversal from positive to negative at the reference time.

Second case has more conductive IP body than the background ($\sigma_2 = 10^{2}\sigma_1$). As shown in Figure \ref{F:IPrespCurves}b, we observe negative transients after $~$1 ms. Compared to canonical model, fundamental response has greater amplitude in whole time range. The maximum amplitude of $d^{F}$ is about $10^{1}$ times greater than canonical case.  In addition, $\dip$ response is greater than that of canonical case. The maximum $\dip$ is about $10^{3}$ times greater than canonical case.  The reference time occur at earlier time ($~$0.05ms) than canonical case ($~$0.1 ms), and the gradients of $\dip$ on charging and discharging are much greater than canonical case. Similar to canonical case, $\frac{\partial b^{IP}_z}{\partial t}$ response show sign reversal at the reference time. 

Third and last case is more resistive IP body than the background ($\sigma_2 = 10^{-2}\sigma_1$). Similar to canonical case, the observed data does not have any negative values. The amplitude of fundamental response is almost same as canonical case. However, $\dip$ response show different phenomenon from previous cases. $\dip$ has positive values at early time, and shows sign reversal at ~0.1ms. The maximum amplitude of negative $\dip$ occurs at ~0.4 ms. At this moment, we chose the reference time in negative $\dip$ values based on the background knowledge that the IP current has reversed direction to the reference current. However, for resistive case, there might be another level of complexity that we need to investigate based on the positive $\dip$ response at early time. We will clarify this phenomenon later by investigating IP currents in the earth. Similar to previous two cases, $\frac{\partial b^{IP}_z}{\partial t}$ response show sign reversal at the reference time. 

While analysis of time decaying curves at one sounding location provided time behaviour of ATEM-IP response, this lacks to recognize the geometric features of IP anomaly from all sounding locations. In order to investigate geometric features of $\dip$ response, we show maps of $d$, $d^F$ and $\dip$ responses at early and late times from all sounding locations.  Date type here is $b_z$ (nT). Left ($d$), middle ($d^F$) and right ($\dip$) panels of Figure \ref{F:IPresp1}a shows maps for canonical model at the early time when t=0.03 ms. Because canonical model does not have any conductivity contrast, $\d$ and $d^F$ at this early time does not show anomalous response. Although $\dip$ response exists in the observed data, relative strength of $\dip$ to $d^F$ is much smaller. Figure \ref{F:IPresp1}b show three responses for conductive case. Due to  induced vortex current, which increases total current, in the the conductive body, we can observe higher anomaly both on $d$ and $d^F$ at this early time. Compared to canonical case, $\dip$ map for conductive case shows more compact distribution. Figure \ref{F:IPresp1}c show three responses for resistive case. Due to the galvanic current, which reduces total current, both $d$ and $d^{F}$ have lower anomaly. This anomalous response compared to conductive case show more spreaded distribution. Different from both canonical and conductive cases, $\dip$ response for this case shows positive anomaly, which is consistent from the observation from transient curves. For all three cases at this early time, $d^F$ is much greater than $\dip$. 

For the later time when $t$ = 7.08 ms, we provided same maps in Figure \ref{F:IPresp2}. At this time $\dip$ is greater than $\d^{F}$ for all three cases. Although we do not observe any negative values on both canonical and resistive cases, by the EM decoupling process we can identify IP response embedded in the observation. Anomalous $\dip$ response for both canonical and resistive cases show more spreaded distribution than conductive case. 

To summarize, the conductive case has the strongest IP signal due to the greater $\d^{F}$ compared to other cases. $\dip$ response for $b_z$ field has minus sign, and has charging and discharging phase. For resistive case, $\dip$ shows positive values at early times, and it needs to be illuminated. At certain later time, IP is considerable to EM effect so that we have some chances to proceed EM decoupling process. At the reference time when we may have maximum polarization charge build-up, $\frac{\partial b^{IP}_z}{\partial t}$ shows zero-crossing where sign reversal occurs for all three cases. $\dip$ response from conductive case show more compact distribution on map view compared to canonical and resistive cases. 

\begin{figure}[htb]
\includegraphics[width=1.0\textwidth]{figures/threecasesresp/EMIP_threecases.png}
  \centering
  \caption{Time decaying curves of $d$ (black line), $d^F$ (blue line) and $\dip$ (red line) responses for three cases: (a) canonical, (b) conductive and (c) resistive. Right and left panels show $b_z$ and $\frac{\partial b_z}{\partial t}$. }
  \label{F:IPrespCurves}
\end{figure}


\begin{figure}[htb]
  \centering  \includegraphics[width=1.0\textwidth]{figures/threecasesresp/IPresp_ch6.png}
  \caption{Response map of $d$ (left panel), $d^{F}$ (middle panel) and $d^{IP}$ (right panel) at early time. Three maps for  (a) Canonical, (b) conductive and (c) resistive cases are shown at 0.03 ms.}
  \label{F:IPresp1}
\end{figure}

\begin{figure}[htb]
  \centering  \includegraphics[width=1.0\textwidth]{figures/threecasesresp/IPresp_ch38.png}
  \caption{Response map of $d$ (left panel), $d^{F}$ (middle panel) and $d^{IP}$ (right panel) at early time. Three maps for  (a) Canonical, (b) conductive and (c) resistive cases are shown at 7.08 ms.}
  \label{F:IPresp2}
\end{figure}
\clearpage

%Seog commenti: This mainly because galvanic current is the main source of the IP effects for both canonical and resistive cases, while for conductive case vortex current is the main source of the IP effect. 


\subsection{Analyses of IP currents}
In previous section, we analyzed IP responses for three different conductivity models: canonical, conductive and resistive. Although we systematically analyzed, we have not clearly showed why $\dip$ response has different characteristics from different conductivity structures: a) $\dip$ response for resistive case has positive values at early time. b) $\dip$ response for conductive case on map view has more compact anomaly. These two questions possibly related to different types of IP currents: galvanic (charge build-up) and vortex (EM induction) currents. Because these two currents have different geometric shape, we can easily recognize each effect on IP current by observing current density in the earth. Both currents will generate magnetic field at receiver locations. We investigate total, fundamental and IP currents where transmitter loop is located at (30 m, -200 m, 0 m). 

Figure \ref{F:IPcurrents1} shows plan view maps of total (left panel), fundamental (middle panel) and IP (right panel) currents at early time (0.03 ms).  Those currents for three different cases: (a) canonical, (b) conductive and (c) resistive are shown. From the equation \ref{eq: jip_approx}, we can expect that amplitude of the IP current may proportional to that of fundamental currents. Direction of the IP current may be opposite to fundamental currents. Three currents for canonical case at the early time show similar IP currents that we expected. From the geometric shape of the IP current we recognize that the major effect of the IP current for this case is due to galvanic IP. For conductive case, we observe more localized fundamental current distribution in the conductive body. Vortex current generated in the conductive body has the same direction to fundamental current thus, this induced current increases total current.  The IP current for this case show more localized distribution in the conductive IP body than canonical case, whereas the direction of the IP currents for both cases is similar at least have opposite directions to the fundamental currents. Similarly, galvanic IP effect is dominant in conductive case. Different from conductive case, fundamental current for resistive case show reduced amplitude due to the resistive body. Furthemore, the IP current has higher amplitude outside of the body, and the direction of the IP current has the opposite directions from those of canonical and conductive cases. Especially near the transmitter location, which is outside of the body, we recognize vortex IP current in the direction of the fundamental current thus, the IP current increases the total current. This observation shows the reason why we have positive $\dip$ response for resistive case at the early time. For resistive case, both galvanic and vortex IP currents has considerable amplitude. At this early time, IP currents for all three cases show two common features: a) amplitude of the IP current is much less than the fundamental current. b) galvanic IP effect is dominant in the IP current. 

Figure \ref{F:IPcurrents2}, show the same three currents on plan view maps, but at the late time (7.08 ms). For all three cases, at this time, the IP currents are dominant compared to the fundamental currents. Here we recognize natural separation of EM and IP effects in time: EM effect is dominant in early time but decays faster than IP effect thus, we observe dominant IP effect at certain late time. Considering this, we recognize the effective time range of EM decoupling will be when the relative strength of IP effect is considerable to EM effect. Although it depends upon the conductivity structure and IP parameters of the earth, but this may commonly happen at certain late time channels in practice. Geometric shape of IP currents for canonical model did not change significantly, whereas that of amplitude decays considerably. In addition, the galvanic IP effect is dominant even at this late time. This shows the similarity with EIP case because the geometric  shape of the IP current will not change for EIP case although it decays in time. In contrast, the IP current for the conductive case show different characteristic. Different from the early time, in the body, we observe dominant vortex IP current in late time. The IP current for conductive case has much localized distribution compared to the IP current for canonical one. Geometric shape of the IP current for the resistive case at this late time significantly changed. Even the direction of the IP current at this late is opposite to that at the early time. This clearly shows the sign reversal in the IP responses for this case. Similar to the early time both galvanic and vortex IP current has considerable magnitudes, and the amplitude of the IP current in the body is much smaller than that in the outside of the body. 

In summary, due to the natural separation of EM and IP effects in time we can expect considerable IP effect in certain late time, and this is the time when we have some possibilities to recognize IP responses embedded in the observation. For canonical model, galvanic IP effect is the main driving force of the IP current. The IP current for conductive case shows transition from galvanic to vortex currents as time increases. Both galvanic and vortex IP currents has considerable amplitude for the resistive case. Conductive case show the strongest amplitude of the IP current. Geometric shape of the IP currents for canonical case does not change significantly from early to late time, whereas those for conductive and resistive cases show considerable changes in time. 

\begin{figure}[htb]
  \centering  \includegraphics[width=1.0\textwidth]{figures/threecasesresp/IPcurrents_ch6.png}
  \caption{Maps of total, fundamental and IP currents at early time. Three maps for  (a) Canonical, (b) conductive and (c) resistive cases are shown at 7.08 ms.}
  \label{F:IPcurrents1}
\end{figure}
\begin{figure}[htb]
  \centering  \includegraphics[width=1.0\textwidth]{figures/threecasesresp/IPcurrents_ch38.png}
  \caption{Maps of total, fundamental and IP currents at early time. Three maps for  (a) Canonical, (b) conductive and (c) resistive cases are shown at 7.08 ms.}
  \label{F:IPcurrents2}
\end{figure}

\subsection{Validation of linearization}
\subsubsection{IP current}
To obtain a linearized kernel in equations \ref{eq: BiotbIP_approx}, we approximate the IP current as a function of pseudo-chargeability. Then by using Biot-Savart law, we evaluate $\dip$ response at receiver location. To show the applicability of our approximate solution, we compare both the IP current and $\dip$ response for the approximate solution with those for the true one. We use the same three conductivity structures to investigate these comparisons. To compute linearized kernel function, we need to choose the time when the amplitude of fundamental current reaches to the maximum for each pixel in 3D. This time and the current were called the maximum time ($t^{max}$) and current ($\j^{F}_{max}$). For all three conductivity models, we computed these $\j^{F}_{max}$ and $t^{max}$ (Figure \ref{F:jmaxtmax}). The maximum current for canonical case show circular current, which is centered at the transmitter location. The maximum time for this case increases as the horizontal distance from the transmitter location increases. This makes sense because a closer pixel from the transmitter location will reach the maximum before a farther pixel does. Due to the conductivity contrasts in conductive and resistive model, $\j^{F}_{max}$ and $t^{max}$ show different features from those for canonica case. The maximum current for conductive case shows the increased amplitude near the transmitter location. In addition, we observe vortex current in the body due to the induced vortex current in the conductive body. This effect is clearly shown in $t^{max}$: in the body, the current reaches to the maximum at later time. In contrast, for resistive case, the amplitude of $\j^{F}_{max}$ is decreased both near the transmitter and in the body. Because of its resistive nature, the current reaches to the maximum time at earlier time in the body. Considering the assumption we made for the polarization current: $\j^{pol}(t) = -\j^F_{max}\peta(t)$, we can recognize that the effect of different conductivity structure in the IP current can be captured through $\j^{F}_{max}$. 

Using these maximum currents for each case with the equation (\ref{eq: jip_approx}), we can compute the approximate IP current ($\j^{IP}_{approx}$). We compare true and approximate $\j^{IP}$. Figures \ref{F:jIPcomparison_early} and \ref{F:jIPcomparison_early} show the comparisons of the true and approximate IP currents on plan view maps at the early (0.03 ms) and late (7.08 ms) times, respectively.  In addition, we plotted time decaying curves of total, fundamental and those IP currents in $y$-direction at the single pixel in the body (noted as Rx in the figures) on Figure \ref{F:jIPcomparison_decay} for canonical (a), conductive (b) and resistive (c) models. Thick dashed and solid lines indicate the early and late times used on above plan view maps.  True and approximate IP currents for canonical case almost coincident not only on plan view maps at this late time (7.08 ms), but also on the early time as shown in Figures (\ref{F:jIPcomparison_early}) and (\ref{F:jIPcomparison_late}). Furthermore, the time decaying curves of these IP currents are almost coincident on all times as shown in Figure (\ref{F:jIPcomparison_decay}). Thus, for the canonical case, our assumptions on the IP current is reasonable for all time range. For conductive case, at the late time,  true and approximate IP currents show reasonable match, whereas they show significant difference at the early time. The approximate IP current for conductive case converges to the true one near 2 ms (Figure (\ref{F:jIPcomparison_decay}(b))) thus, our approximate solution is reasonable near this time for conductive case. Different from two previous cases, the IP current for resistive case have greater amplitude on the outside of the body. At the early time, the approximate solution show significant difference from true one. Even at the late time, the amplitude of true and approximate IP currents show considerable difference on the outside of the body. However, those for inside of the body show reasonable match. In addition, the direction of the true and approximate IP currents are almost identical at the late time. We summarized above comparisons in Table (\ref{F:tableIPcurrent}). A principal assumption that we made for the polarization current was $\j^{pol}(t) \approx \j^{F}_{max}\peta(t)$. The observation that the approximate IP current in the body show good approximation for all three cases at late time suggests that the above approximation is reasonable in the late time. At the late time the approximate IP current show good match with true one for all three cases except for the amplitude of resistive case. This makes the applicability of the linearized kernel function somewhat arguable for the resistive case. 

\begin{figure}[htb]
  \centering  \includegraphics[width=1.0\textwidth]{figures/threecasesresp/jmax.png}\\
  \centering  \includegraphics[width=1.0\textwidth]{figures/threecasesresp/tmax.png}
  \caption{Plan view maps of $\j^{F}_{max}$ (a) and $t^{max}$ (b) at -125 m depth. Left, middle and right panels correspondingly show those for canonical, conductive and resistive models.}
  \label{F:jmaxtmax}
\end{figure}

\begin{figure}[htb]
\centering  
    \includegraphics[width=1.0\textwidth]{figures/threecasesresp/jIPcomparison_ch6.png}
    \caption{Plan view maps of the true (a) and approximate (b) $\j^{IP}$ at -125 m depth and t=0.03 ms. Left, middle and right panels correspondingly show those for canonical, conductive and resistive models.}
  \label{F:jIPcomparison_early}
\end{figure}

\begin{figure}[htb]
  \centering  \includegraphics[width=1.0\textwidth]{figures/threecasesresp/jIPcomparison_ch38.png}
  \caption{Plan view maps of the true (a) and approximate (b) $\j^{IP}$ at -125 m depth and t=7.08 ms. Left, middle and right panels correspondingly show those for canonical, conductive and resistive models.}
  \label{F:jIPcomparison_late}
\end{figure}


\begin{figure}[htb]
  \centering  \includegraphics[width=1.0\textwidth]{figures/threecasesresp/jIPcomparison_decay.png}\\
  \caption{Plan view maps of the true (a) and approximate (b) $\j^{IP}$ at -125 m depth. Left, middle and right panels correspondingly show those for canonical, conductive and resistive models.}
  \label{F:jIPcomparison_decay}
\end{figure}

\begin{figure}[htb]
  \centering  \includegraphics[width=1.0\textwidth]{figures/threecasesresp/tableIPcurrent.png}
  \caption{Comparisons of the applicability of the approximate IP currents for three different cases: canonical, conductive and resistive conductivity models. }
  \label{F:tableIPcurrent}
\end{figure}

\clearpage

\subsubsection{IP response}
Although the IP current is the core of the linearized kernel function, the output of this function is $\dip$ response, which was evaluated by using Biot-Savart law with the approximate IP current. In order to validate the linearized kernel, we also need to compare $\dip$ response generated by the linearized kernel function with the true $\dip$ response, which is computed by subtraction process shown in equation (\ref{eq: IPdatum_syn}). Because we use the coincident loopy system the receiver location is same as transmitter location. The data type we chose in our example is a vertical component of magnetic flux density ($b^{IP}_z$). To show the reliability of applying Biot-Savart law, we also plotted $b^{IP}_z$ response computed by using Biot-Savart law ($b^{IP}_{z \ BS}$) to the true IP current. As shown in Figure \ref{F:compIPresp}, they are almost identical to true $b^{IP}_z$. In the same figure, we compare true and approximate $\dip$ responses for three cases. True and approximate IP responses for the canonical model  (black) shows good match on all times except for very early time (~0.01 ms). The approximate IP response for the conductive case (blue)  converges to the true one as we go to the later time, whereas they show significant difference on early time. This is consistent result with the phenomenon on the approximate IP current of conductive case. True and approximate IP responses for the resistive case show considerable difference for all times due to the poor amplitude of the approximate IP current. 

\begin{figure}[htb]
  \centering  \includegraphics[width=1.0\textwidth]{figures/threecasesresp/compIPresp.png}
  \caption{Comparison of of true and approximate $\dip$ responses. The data type here is vertical component of the magnetic flux density ($b_z$). Black, blue and red color correspondingly stand for canonical, conductive and resistive cases. }
  \label{F:compIPresp}
\end{figure}
\clearpage

\subsubsection{Discussion for the resistive case}
Comparisons of true and approximate solutions for the IP current and response suggest that our linearized kernel function is reasonable at certain late time except for the resistive case. This is mostly resulted from the poor approximation for the amplitude of the IP current. Then why the approximate IP current lose the amplitude? We try to explain this phenomenon by decomposing the IP current as three parts:
\begin{equation}
  \j^{IP} = \j^{pol} -\vec{a}^{IP}_J-\grad \phi^{IP}_J, 
\end{equation}
where $\siginf\e^{IP} = -\vec{a}^{IP}_J - \grad \phi^{IP}_J$. This is Helmholtz decomposition of $\siginf\e^{IP}$ thus, $\vec{a}^{IP}_J = 0$. Because we recognized that our approximation for  $\j^{pol}$ is good, possible source of the reason why our assumption does not work well for the resistive case is $\siginf \e^{IP}$. In addition, considering Biot-savart law shown in equation (\ref{eq: Biot}), computed $\b^{IP}$ does not affected by $\grad \phi^{IP}_J$ because it is curl-free component. Therefore, if the relative strength of $\vec{a}^{IP}_J$ is greater than $\j^{pol}$, the approximate IP current may not describe the amplitude of the true IP current and this will induce same effect on the approximate $\dip$ response. Figure \ref{F:jpolvsj1IP} show $\j^{pol}$, $-\vec{a}^{IP}_J$ and $-\grad \phi^{IP}_J$ for all three cases at 7.08 ms. The amplitude of the polarization current for canonical and conductive cases at this late time are greater than those of  $\vec{a}^{IP}_J$ and $\grad \phi^{IP}_J$. However, in the resistive case, the amplitude of $\vec{a}^{IP}_J$ is greater than that of $\j^{pol}$. This shows the reason why our approximation has poor description for the amplitude of the true IP current and $\dip$ response in the resistive case. 

\begin{figure}[htb]
  \centering  \includegraphics[width=1.0\textwidth]{figures/threecasesresp/jpolvsj1IP_ch38.png}
  \caption{Plan view maps of the true (a) and approximate (b) $\j^{IP}$ at -125 m depth and t=7.08 ms. Left, middle and right panels correspondingly show those for canonical, conductive and resistive models.}
  \label{F:jpolvsj1IP}
\end{figure}
\clearpage
%%===================================================================
\subsection{A potential for recovering IP parameters}

Although the physical understanding of the linearized kernel function is important, our final goal will be the IP inversion of the ATEM data. To investigate a potential of this task, we extract intrinsic IP parameters from a single pixel of the pseudo-chargeability, which is located at the center of the IP body (marked as Rx on Figures \ref{F:jIPcomparison_late}). Recalling that the pseudo-chargeability is the convolution between $\peta^{I}(t)$ and $w^{e}(t)$ (\ref{eq: pseudochargeability}), we need to solve a deconvolution problem to extract intrinsic IP parameters including $\eta$ and $\tau$. Since we consider Debye model ($c=1$) for Cole-Cole model, we assume that we know $c$. Definition of intrinsic pseudo-chargeability is shown in equation (\ref{eq: intrinsic_peta}). We compute $\peta$ and $w^e$ for all three cases to make a synthetic data for this deconvolution problem. Using a conventional deconvolution problem, which directly recover impulse response, we can recover $\peta^{I}(t)$. However, we are more interested in Cole-Cole parameters including $\eta$ and $\tau$. Accordingly, we set these two IP parameters as our inversion model. Using the Gauss-Newton method, we optimize this problem and recover these two parameters. We consider the pseudo-chargeability at a single pixel as the observed data, and convolution between $\peta^{I}(t)$ and $w^e(t)$ as a forward problem. Figure \ref{F:wepetathree}a and b show $w^e$ and comparisons of the observed and predicted data. Because the $w^e(t)$ can be considered as normalized time history of the electric field, for all cases it starts from zero and increases until it reaches to the peak.  After that it decays. Resistive case reaches to the maximum at the earliest time. Canonical and conductive cases reach at the maximum almost same time, although the conductive case show much slower decay after this time. Those phenomenon make sense considering their conductivity features with EM induction. As shown in Figure,  recovered $\tau$ and $\eta$ show are same as true ones, and the observed and predicted data show good matches. The true $\tau$ and $\eta$ for all three cases were 0.005 and 0.2, respectively. Considering the approximate IP currents in the body for all three cases showed good match with the true ones (Figure \ref{F:jIPcomparison_decay}), we can recognize the potential for recovering intrinsic IP parameters from airborne time domain EM data. 

\begin{figure}[htb]
  \centering  \includegraphics[width=1.0\textwidth]{figures/threecasesresp/wepetathree.png}
  \caption{}
  \label{F:wepetathree}
\end{figure}




% \subsection{Linear inversion}
% We apply linear inversion to restore 3D distribution of pseudo-chargeability at each time channels. In order to validate our inversion algorithm, we first generated synthetic IP data, $d^{IP}_{syn}$ for both canonical and conductive models using forward equation shown in equation (\ref{eq: BiotbIPdt_approx}). Floor noise of $0.01max(|\mathbf{d}^{obs}|)$ was added to $\dip_{syn}$, and we set this as our uncertainty ($\epsilon$). Here data type is $\frac{\partial b_z}{\partial t}$. We use same survey configuration that we used to compute $d^{IP}$ responses. As shown in Table 1, the number of data for each time channel is 121, since we have 121 stations. Our model parameter do not include air, thus the number of model parameters: 41$\times$41$\times$20=33620. Since our sensitivity function has higher values near the surface, we need to balance this biased sensitivity distribution. This has been carefully treated in magnetic inversion (CITE). Based on this, we use $\mathbf{w} = (\mathbf{z}+z_0)^{-1.5}$ with $z_0$=0 for model weighting shown in equation (\ref{eq: weight_mat}). Here $\mathbf{z}$ is discretized depth for all model parameters, $z_0$ indicates the depth of the earth surface. Other inversion parameters are shown in Table 1. A constant, related to initial $\beta$, $r_{\beta}$ is set to 1. To generate sensitivity matrix shown in equation (\ref{eq: dbIPdt_linear}), we used true background conductivity model for both cases. Same Cole-Cole parameters are used ($\eta$=0.2, $\tau$=0.005 and $c$=1), and for $\peta$ we used $\peta^{I}$, which was shown in equation (\ref{eq: intrinsic_peta}) when $c$=1.

% \begin{table}[ht]
% \caption{Parameters used in Projected Gauss Newton inversion.} % title of Table
% \centering % used for centering table
% \begin{tabular}{c c} % centered columns (4 columns)
% \hline\hline %inserts double horizontal lines
% Inversion parameters & Value \\
% [0.1ex] % inserts table
% \hline
% The number of data &    11$\times$11=121 \\
% The number of models &  41$\times$41$\times$20=33620 \\
% $\alpha_s$ &  $10^{-5}$\\
% $\alpha_x$ &  1\\
% $\alpha_y$ &  1\\
% $\alpha_z$ &  1\\
% Initial $\beta$ &  $r_{\beta}\frac{\mathbf{x}^T\mathbf{J}^T\mathbf{J}\mathbf{x}}{\mathbf{x}^T\mathbf{W}_m^T\mathbf{W}_m\mathbf{x}}$\\
% Decrease rate of $\beta$ &  0.5\\
% The number of $\beta$ iteration & 2\\
% $\mathbf{m}_{ref}$ & $10^{-10}$ \\
% Uncertainty & $\epsilon$=$0.01$max($|\mathbf{d}^{obs}|$) \\
% Bound constraint & $\mathbf{m} > 0$\\
% \hline %inserts single line
% \end{tabular}
% \label{table:1} % is used to refer this table in the text
% \end{table}

% Linear inversions are applied $d^{IP}_{syn}$ data at t=4.7 ms so that corresponding $\peta^I$ value for IP body is 8.52. In Figure ~\ref{F: Peta_dipsyn_ch38}a and b, we presented inverted peudo-chargeablity models for both canonical and conductive models, respectively. To show reliability of our inversion, observed and predicted data for both canonical and conductive model were shown in Figure ~\ref{F: ObsPred_syn_ch38}a and b, respectively. Observed and predicted data for both cases show reasonable matches. Also, both inversions were reached to th target misfit. In both cases, geometry of IP body was well imaged in right location, whereas estimated $\peta$ for canonical model shows somewhat blurred image compared to that for conductive model. Recovered $\peta$ values in the IP body for both cases were $~$1.5 and $~$4.4, whereas true $\peta$ value for the IP body was 8.72. As Figure ~\ref{F: ObsPred_syn_ch38}a and b, $d^{IP}_{syn}$ for canonical model shows broader distribution than that for conductive model, which may explain different magnitude of restored $\peta$ for both cases. These observations of recovered $\peta$ for both cases shows that recovered model from conductive model can have bigger magnitude compared to canonical model. In addition, by comparing $d^{IP}_{syn}$ shown in the left panel of Figure ~\ref{F: ObsPred_syn_ch38}a and b with true $\dip$ data shown in the right panels of Figures ~\ref{F: EMIPresp2_case1}b and ~\ref{F: EMIPresp2_case2}b, we recognize that their geometric distribution for both cases are analogous, whereas their absolute values are different. This suggest the reliability of our linearization approach.

% \begin{figure}[htb]
%   \centering
%   \includegraphics[width=\textwidth]{figures/synthetic/PetaCase1_syn_ch38.png}\\
%   (a)
%   \includegraphics[width=\textwidth]{figures/synthetic/PetaCase2_syn_ch38.png}\\
%   (b)
%   \caption{Slices of estimated $\peta$ distributions at $t = $ 4.7 ms by inverting $d^{IP}_{syn}$. (a) Canonical model. (b)  Conductive model. Left and right panel show plan and section view at -125 m depth and 0 m northing. Black dashed lines outline IP body. }
%   \label{F: Peta_dipsyn_ch38}
% \end{figure}

% \begin{figure}[htb]
%   \centering
%   \includegraphics[width=\textwidth]{figures/synthetic/ObsPred_syn_ch38_case1.png}\\
%   (a)
%   \includegraphics[width=\textwidth]{figures/synthetic/ObsPred_syn_ch38_case2.png}\\
%   (b)
%   \caption{Slices of estimated $\peta$ distributions at $t = $ 4.7 ms by inverting $d^{IP}_{syn}$. (a) Canonical model. (b)  Conductive model. Left and right panel show plan and section view at -125 m depth and 0 m northing. Black dashed lines outline IP body. }
%   \label{F: ObsPred_syn_ch38}
% \end{figure}

% On the other hand, we consider true $d^{IP}$ data at t=4.7 ms, which was generated by subtracting $d^{F}$ from $d$ as shown in the right panels of Figures ~\ref{F: EMIPresp2_case1}b and ~\ref{F: EMIPresp2_case2}b. Since $\peta$ is convoluted property of electric field and $\peta^{I}$, we need to evaluate restored $\peta$ can suggest meaningful information about the embedded IP body. Same survey configuration and inversion parameters were used as above examples for $d^{IP}_{syn}$ except $r_{\beta}$=0.1. In Figure ~\ref{F: Peta_dip_ch38}a and b, we presented restored pseudo-chargeability models for both canonical and conductive models, respectively. Both inversions for canonical and conductive models were reached to the target misfit. Recovered $\peta$ models for both cases show reasonable geometry of IP body, whereas absolute magnitude of them are quite different from $\peta^{I}$. However, $\peta$ is convoluted property of $\e(t)$ and $\peta^{I}$ so that absolute value of restored $\peta$ does not have meaningful information. Therefore, restored $\peta$ can suggest geometry of IP body in the earth. Similar patterns from recovered $\peta$ models of $\dip_{syn}$ can be observed here: restored model from canonical model show broader distribution and smaller magnitude of $\peta$ than that from conductive model. Accordingly, identification of relative strength of $\peta^I$ for different IP bodies might depend on background conductivity model. In addition, to investigate effect of wrong background model, we invert $d^{IP}$ data for conductive model using sensitivity matrix generated for canonical model. Restored $\peta$ distribution is shown in Figure ~\ref{F: Peta_case2_wrong_ch38} and it is almost same as Figure ~\ref{F: Peta_dip_ch38}b, which used correct sensitivity matrix. This shows the robustness of our inversion to background conductivity model. However, note that this effect in EM decoupling process can be more significant.

% \begin{figure}[htb]
%   \centering
%   \includegraphics[width=\textwidth]{figures/synthetic/PetaCase1_true_ch38.png}\\
%   (a)
%   \includegraphics[width=\textwidth]{figures/synthetic/PetaCase2_true_ch38.png}\\
%   (b)
%   \caption{Slices of estimated $\peta$ distributions at $t = $ 4.7 ms  by inverting $d^{IP}$. (a) Canonical model. (b)  Conductive model.. Left and right panel show plan and section view at -125 m depth and 0 m northing. }
%   \label{F: Peta_dip_ch38}
% \end{figure}

% \begin{figure}[htb]
%   \centering
%   \includegraphics[width=\textwidth]{figures/synthetic/PetaCase2_wrong_ch38.png}
%   \caption{Slices of estimated $\peta$ distributions at $t = $ 4.7 ms for conductive model with incorrect background conductivity model. Left and right panel show plan and section view at -125 m depth and 0 m northing. }
%   \label{F: Peta_case2_wrong_ch38}
% \end{figure}

% Finally, using same survey and inversion set-ups, we inverted 14 time channels of $\d^{IP}$ ranging from 1 to 10 ms for both cases. We presented time decaying curves of restored $\peta$ models at (0, 0, -125) for both cases in Figure ~\ref{F: Peta_Center}a and b. Since we already recognized absolute value of the estimated pseudo-chargeability, $\peta_{est}$ is not meaningful, significant magnitude difference between $\peta^{est}$ and $\peta^I$ makes sense. However, we identify slope of $\peta^{est}$ and $\peta^I$ looks similar in later time channels (2-10 ms) for both cases. Taking log to equation (\ref{eq: intrinsic_peta}) yields
% \begin{equation}
%     log(\peta^{I}) = log(\frac{\eta}{(1-\eta)\tau}) - \frac{1}{(1-\eta)\tau}t,
% \end{equation}
% which can be considered as linear equation. To estimate slope of the above equation, $- \frac{1}{(1-\eta)\tau}$, we fit active set of the recovered pseudo chageability, $\peta^{est}_{active}$ ranging from 2.5 ms to 10 ms. Estimated slopes for canonical and conductive models are -203 and -218 when the true slope is -250. Corresponding relative errors are 12 and 13$\%$. This is promising result, which suggests potential of restoring intrinsic Cole-Cole parameters from recovered $\peta$ model at each time channel.

% \begin{figure}[htb]
%   \centering
%   \includegraphics[width=0.8\textwidth]{figures/synthetic/PetaCase1_center.png} \\(a)\\
%   \includegraphics[width=0.8\textwidth]{figures/synthetic/PetaCase2_center.png} \\(b)
%   \caption{Time decaying curves of recovered and intrinsic pseudo chargeability at (0, 0, -125). (a) Canonical model. (b) Conductive model. Black dashed and solid line with solid circle indicate $\peta^{I}$ and $\peta^{est}$. Empty circles indicate active set of $\peta^{est}$, which are used to compute slope of time decaying $\peta^{est}$ in semi-log plot. }
%   \label{F: Peta_Center}
% \end{figure}

% \clearpage
% \subsection{Field example: Mt. Milligan}
% % \subsection{EIP}
% % \subsection{MIP}
% % \subsection{EM-IP with grounded source}
% % \subsection{EM-IP with inductive source (ATEM) }

% \clearpage
% \section{Appendix}
% \subsection{Derivation of the sensitivity   in discretized space}
% The numerical evaluation of Maxwell's equations for the steady-state case can be derived by discretizing the system in space using finite volume (FV) method with weak formulation (CITE). For the discretization, we assume that the electric field $\e$ is discretized by grid function $\de$ on cell edges and magnetic flux density $\b$ is discretized by grid fuction $\db$ on cell faces. Electrical potential $\phi$ is discretized by grid fucntion  $\phi$ on cell nodes. For clear representation of the derivation, recall Maxwell's equations in steady state as
% \begin{align}
%   \j = \sigma\e = -\sigma\grad \phi, \\
%   -\div \j = \div \j_s, \\
%   \j\big|_{\partial \Omega}\cdot\hat{n} = 0,
%   \label{eq:DCBCneumann}
% \end{align}
% where $\partial \Omega$ indicates boundary surface of the system and $\hat{n}$ is the normal vector of the boundary surface. Modified from CITE, weak form of those equations can be written as
% \begin{align}
%   (\j, \vec{w}) + (\sigma \grad \phi, \vec{w}) = 0, \\
%   -(\j, \grad \psi) = (\j_s, \grad \psi).
% \end{align}
% The inner products $(\j, \vec{w})$, $(\sigma \grad \phi, \vec{w})$,  $(\j, \grad \psi)$ and $(\j_s, \grad \psi)$ are edge based products. Here we define the inner product as
% \begin{equation}
%   (\vec{a}, \vec{b}) = \int_{\Omega} \vec{a}\cdot\vec{b} dv,
% \end{equation}
% where $\Omega$ is the volume of the system. By discretizing $\grad$ operator and the inner product in space, we obtain
% \begin{equation}
%   \Me\dj + \MeSig\dgrad\phi = 0,
%   \label{eq:DCdisceq1}
% \end{equation}
% \begin{equation}
%   -\dgrad^T \Me\dj = \dgrad^T \Me\dj_s,
%   \label{eq:DCdisceq2}
% \end{equation}
% where $\mathbf{M}^e_i$ is the mass matrices, which discretize the edge based inner product (CITE). This inner products are defined  as
% \begin{align}
%   \mathbf{M}^e_i = \diag(\Ace^T\diag(\vol)\mathbf{i}).
% \end{align}
% Here, $\mathbf{i}$ indicates a grid function on cell center like $\sigma$, and $\vol$ is the grid function for the cell volume. The averaging matrix $\Ace$ averages discrete function defined on the edges to the cell center. The mass matrix $\Me$ without subscript $i$ indicates that $\mathbf{i}$ is equal to the identity column vector of which all elements are one. By substituting equation (\ref{eq:DCdisceq1}) to (\ref{eq:DCdisceq2}), we have
% \begin{equation}
%   \A^{DC}\phi = \mathbf{rhs}^{DC},
%   \label{eq:DCdiscLin}
% \end{equation}
% where $\A^{DC} = \dgrad^T \MeSig\dgrad$ and $\mathbf{rhs}^{DC} = \dgrad^T \Me\dj_s$. Sensitivity function of $\phi$ can be derived by taking derivative to equation (\ref{eq:DCdiscLin}) in terms of $\sigma$:
% \begin{equation}
%   \frac{\partial \phi}{\partial \sigma} = -(\A^{DC})^{-1}\dgrad^T\diag(\dgrad\phi)\Ace^T\diag(\vol).
%   \label{eq:DCsensedisc}
% \end{equation}
% We recall that continuous sensitivity function shown in equation (\ref{eq: senseDC}), which can be modified as
% \begin{equation}
%   \frac{\partial \phi}{\partial \sigma} = -[\div\sigma\grad]^{-1}\div\grad\phi.
%   \label{eq:DCsensecont}
% \end{equation}
% One can intuitively identify that continuous sensitivity function shown in equation (\ref{eq:DCsensecont}) can be discretized as equation (\ref{eq:DCsensedisc}) with the boundary condition specified on equation (\ref{eq:DCBCneumann}). This approach shows that some advantages of working in discretized space even for the derivation of continuous function (CITE).


\clearpage
\bibliographystyle{plain}
\bibliography{reference}

\end{document}



% http://www.math.mun.ca/tex-archive/macros/latex/contrib/todonotes/todonotes.pdf
